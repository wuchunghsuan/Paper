%\documentclass[a4paper,fleqn,longmktitle]{cas-dc}
\documentclass[a4paper,fleqn]{cas-dc}

\usepackage[numbers]{natbib}
\usepackage{lipsum}
\usepackage{subcaption}
\usepackage{algpseudocode}
\usepackage{algorithm}
\usepackage{enumitem}
\usepackage{booktabs}
\usepackage{float}
\usepackage{color}
\usepackage{hyperref}
\usepackage{amsmath}
\usepackage{graphics}
\usepackage{multicol}
\usepackage{varwidth}

% for revision
\newcommand{\edited}[1]{{\color{black}{}#1}}
\newcommand{\reedited}[1]{{\color{blue}{}#1}}

% \xspaceaddexceptions{]}
\def\tsc#1{\csdef{#1}{\textsc{\lowercase{#1}}\xspace}}
\tsc{WGM}
\tsc{QE}
\tsc{EP}
\tsc{PMS}
\tsc{BEC}
\tsc{DE}

\begin{document}
\let\WriteBookmarks\relax
\def\floatpagepagefraction{1}
\def\textpagefraction{.001}
\shorttitle{Efficient Shuffle Management for DAG Computing Frameworks Based on the FRQ Model}
\shortauthors{R. Rui et al.}
%\begin{frontmatter}

\title [mode = title]{Efficient Shuffle Management for DAG Computing Frameworks Based on the FRQ Model}
% \tnotemark[1,2]

%%  \tnoteref{tn1,tn2}

% \tnotetext[1]{This document is the results of the research
%    project funded by the National Science Foundation.}

% \tnotetext[2]{The second title footnote which is a longer text matter
%    to fill through the whole text width and overflow into
%    another line in the footnotes area of the first page.}

\author{Rui Ren}[type=editor,
                        bioid=1,
                        style=chinese]
% \fnmark[1]

\author{Chunghsuan Wu}[style=chinese]
\author{Zhouwang Fu}[style=chinese]
\author{Tao Song}[style=chinese]
\author{Yanqiang Liu}[style=chinese, orcid=0000-0002-8343-7923]
\cormark[1]
\ead{oai@sjtu.edu.cn}
\author{Zhengwei Qi}[style=chinese]
\author{Haibing Guan}[style=chinese]

% \address[1]{Shanghai Jiao Tong University, China}
\cortext[cor1]{Corresponding author}

\begin{abstract}

In large-scale data-parallel analytics, shuffle, or the cross-network read and aggregation of partitioned data between tasks with data dependencies, usually brings in large overhead. Due to the dependency constrains, execution of those descendant tasks could be delayed by logy shuffles.
To reduce shuffle overhead, we present \textit{SCache}, an open source plug-in system that particularly focuses on shuffle optimization. 
{\color{blue}
During shuffle optimization process, we propose a performance model called \textit{Framework Resources Quantification} (FRQ) model. 
FRQ model displays utilization of resources in time dimension and calculate execution time of computing jobs on different DAG frameworks. 
We use FRQ model to assist in analyzing shuffle process and verify SCache shuffle optimization by mathematics.
}
By extracting and analyzing shuffle dependencies prior to the actual task execution, SCache can adopt heuristic pre-scheduling combining with shuffle size prediction to pre-fetch shuffle data and balance load on each node. 
Meanwhile, SCache takes full advantage of the system memory to accelerate the shuffle process. 
% {\color{red}
% We have implemented SCache and customized Spark to use it as the external shuffle service and co-scheduler. The performance of SCache is evaluated with both simulations and testbed experiments on a 50-node Amazon EC2 cluster.
% Those evaluations have demonstrated that, by incorporating SCache, the shuffle overhead of Spark can be reduced by nearly $89\%$, and the overall completion time of TPC-DS queries improves $40\%$ on average.
% }
{\color{blue}
We have implemented SCache as the external shuffle service and co-scheduler on both Apache Spark and Apache Hadoop. The performance of SCache is evaluated with both simulations and testbed experiments on a 50-node Amazon EC2 cluster.
Those evaluations have demonstrated that, by incorporating SCache, the shuffle overhead of Spark can be reduced by nearly $89\%$, and the overall completion time of TPC-DS queries improves $40\%$ on average. On Apache Hadoop, SCache optimize end-to-end Terasort completion time by 15\%.
}
\end{abstract}


\begin{keywords}
  Distributed DAG frameworks \sep Shuffle \sep Optimization \sep Performance model
\end{keywords}

\maketitle

\section{Introduction}\label{sec:introduction}
\IEEEPARstart{W}{e} are in an era of data explosion --- 2.5 quintillion bytes of data are created every day according to IBM's report\footnote{http://www-01.ibm.com/software/data/bigdata/}.
{\color{black}
In \textit{industry 4.0}, an increasing number of IoT sensors are embedded in the industrial production line \cite{lade2017manufacturing}.
During the manufacturing process, the information about the assembly lines, stations, and machines is continuously generated and collected.
Using distributed computing frameworks to analyze industrial big data is an inevitable trend \cite{lee2014service}.
% \cite{ibmreport}.
% These data come from a variety of industry, e.g., records in industrial production, digital pictures and videos in social networks, and sensors used for Internet of Things (IoT).
% These data come from a variety of industries, e.g., e-commerce, telecommunications, media, retail customers and social networks.
% \cite{chen2012interactive}.
% Whether the data is generated from industrial production or social networks, the analysis methods are almost MapReduce-related\cite{lv2017next}.
% Nowadays, the real challenge of big data is not how to collect it, but how to manage it logically and efficiently\cite{lv2017next}.
% Several sophisticated frameworks are used in industrial big data analysis such as Hadoop MapReduce\footnote{http://hadoop.apache.com/}, Spark\footnote{https://spark.apache.org/}, and Storm\footnote{http://storm.apache.com/}.
% The analysis method for these industrial big data mainly relies on MapReduce-liked frameworks, such as Hadoop MapReduce\footnote{http://hadoop.apache.com/}, Spark\cite{spark}, and Storm\footnote{http://storm.apache.com/}.
% Recent years have witnessed the widespread use of sophisticated frameworks, such as Hadoop MapReduce\footnote{http://hadoop.apache.com/}, Dryad \cite{dryad}, Spark \cite{spark}, and Apache Tez \cite{tez}.
Industrial big data is more structured, correlated, and ready for analytics than traditional big data, because industrial big data is generated by automated equipment \cite{basanta2018efficient, lv2017next}.
}

{\color{black}
According to a cross-industry study \cite{chen2012interactive}, both industrial and traditional big data share a characteristic during analytical processing --- a small fraction of the daily workload uses well over 90\% of the cluster’s resources, and these workloads often contain a huge shuffle size. 
According to another MapReduce trace analysis from Facebook, the shuffle phase accounts for $33\%$ of the job completion time on average, and up to $70\%$ in shuffle-heavy jobs \cite{managing}.
The shuffle phase is crucial and heavily affecting the end-to-end application performance.
% Most of the popular frameworks define jobs as directed acyclic graphs (DAGs), such as map-reduce pipeline in Hadoop MapReduce\footnote{http://hadoop.apache.com/}, lineage of resilient distributed datasets (RDDs) in Spark\footnote{https://spark.apache.org/}, vertices and edges in Dryad \cite{dryad}, and Tez\footnote{https://tez.apache.org/}, etc.
Most of the popular frameworks define jobs as directed acyclic graphs (DAGs), such as map-reduce pipeline in Hadoop MapReduce\footnote{http://hadoop.apache.com/}, \textit{RDDs} in Spark\footnote{https://spark.apache.org/}, vertices in Dryad \cite{dryad}, etc.
Shuffle phase is always essential as communication between successive computation stages.
% Despite the differences among data-intensive frameworks, the shuffle phase is always essential as communication between successive computation stages.
}

\begin{figure}
	\centering
	\includegraphics
		[width=.85\linewidth]
		{fig/workflow}
	\caption{Workflow Comparison between Legacy DAG Computing Frameworks and Frameworks with SCache}
	\label{fig:workflow}
\end{figure}
Although continuous efforts of performance optimization have been made among a variety of computing frameworks\cite{chen2017parallel, sync, tachyon, heintz2016end, cheng2017improving, wasi2017comprehensive}, the shuffle phase is often poorly optimized in practice.
In particular, we observe that one major deficiency lies in the coupled scheduling among different system resources.
% As Figure \ref{fig:workflow} shows, the \textit{shuffle write} is responsible for writing intermediate results to disk, which is attached to the tasks in ancestor stages (i.e., map task).  
As Figure \ref{fig:workflow} shows, the \textit{shuffle write} is responsible for writing intermediate results to disk.  
% And the \textit{shuffle read} fetches intermediate results from remote disks through network, which is commonly integrated as part of the tasks in descendant stages (i.e., reduce task). 
And the \textit{shuffle read} fetches intermediate results from remote disks through the network. 
Once scheduled, a fixed bundle of resources (i.e., CPU, memory, disk, and network) named \textit{slot} is assigned to a task, and \textit{slot}s are released only after the task finishes.
Such task aggregation together with the coupled scheduling effectively simplifies task management.
% However, since a cluster has a limited number of slots, attaching the I/O intensive shuffle phase to the CPU/memory intensive computation phase results in a poor multiplexing between computational and I/O resources.
However, attaching the I/O intensive shuffle phase to the CPU/memory intensive computation phase results in a poor multiplexing between computational and I/O resources.
% Moreover, the shuffle read phase introduces all-to-all communication pattern across the network, and such network I/O procedure is also poorly coordinated.
Moreover, since the shuffle read phase starts fetching data only after the corresponding reduce task starts, all the corresponding reduce tasks start fetching shuffle data almost simultaneously.
% Note that the shuffle read phase starts fetching data only after the corresponding reduce task starts.
% Meanwhile, the reduce tasks belonging to the same execution phase are scheduled at the same time by default. 
% As a result, all the corresponding reduce tasks start fetching shuffle data almost simultaneously.
Such synchronized network communication causes a burst demand for network I/O, which in turn greatly enlarges the shuffle read completion time. 
% To desynchronize the network communication, an intuitive way is to launch some tasks in the descendent stage earlier, such as \textit{slow-start} from Hadoop MapReduce.
{\color{black}
% However, such early-start is by no means a panacea.
% This is mainly because the early-start always introduces an extra early allocation of the slot leading to a slow execution of the current stage.
% However, such early-start has several deficiencies.
% First, the early-start always introduces an extra early allocation of the slot leading to a slow execution of the current stage.
% Second, the early-start can not fully overlap the shuffle phases and the descendent stage. Due to the shuffle phases are coupled with the reduce phases and the size of slots is limited, only parts of reduce tasks can be early-start. 
% Last but not least, it is hard to find the optimal time to begin the early-start. If too early, the descendent stages are not ready to output enough intermediate data for the shuffle. If too late, the idle network resource is wasted.
}

% We note that the above deficiencies generally exist in most of the DAG computing frameworks. 
% As a result, even though we can effectively resolve the above deficiencies by modifying one framework, updating one application at a time is impractical given the sheer number of computing frameworks available today.

{\color{black}
To optimize the data shuffling without significantly changing DAG frameworks, we propose S(huffle)Cache, an open source\footnote{https://github.com/frankfzw/SCache} plug-in system for different DAG computing frameworks.
Specifically, SCache takes over the whole shuffle phase from the underlying framework.
The workflow of a DAG framework with SCache is presented in Figure \ref{fig:workflow}. 
SCache replaces the disk operations of shuffle write by the memory copy in map tasks and pre-fetch the shuffle data.
% Can we efficiently optimize the data shuffling without significantly changing DAG frameworks?
% In this paper, we answer this question in the affirmative with S(huffle)Cache, an open source\footnote{https://github.com/frankfzw/SCache} plug-in system which provides a shuffle-specific optimization for different DAG computing frameworks.
% Specifically, SCache takes over the whole shuffle phase from the underlying framework by providing a cross-framework API for both shuffle write and read.
SCache's effectiveness lies in the following two key ideas.
First, SCache decouples the shuffle write and read from both map and reduce tasks.
Such decoupling effectively enables more flexible resource management and better multiplexing between the computational and I/O resources.
Second, SCache pre-schedules the reduce tasks without launching them and pre-fetches the shuffle data. 
Such pre-scheduling and pre-fetching effectively overlap the network transfer time, desynchronize the network communication, 
and avoid the extra early allocation of slots.

We evaluate SCache on a 50-node Amazon EC2 cluster on both Spark and Hadoop MapReduce.
In a nutshell, SCache can eliminate explicit shuffle time by at most $89\%$ in varied applications. More impressively, SCache reduces $~40\%$ of overall completion time of TPC-DS\footnote{http://www.tpc.org/tpcds/}, a standardized industry benchmark, on average on Apache Spark.

The rest of the paper is organized as follows:
{\color{blue}
% We discuss the opportunities to optimize shuffle in Section \ref{motivation}.
We present the methodologies of optimization which using by SCache in Section \ref{opt} and detail the implementation of SCache in Section \ref{impl}.
We introduce the FRQ performance model in Section \ref{model}.
In Section \ref{evaluation}, we present comprehensive evaluations about SCache and the FRQ model.
% We present the implementation of SCache in Section \ref{impl} and do comprehensive evaluations in Section \ref{evaluation}.
We also discuss the related work in Section \ref{related} and conclude in Section \ref{conclusion}.
}
}

% The workflow of a DAG framework with SCache is presented in Figure \ref{fig:workflow}. 
% SCache replaces the disk operations of shuffle write by the memory copy in map tasks. 
% The slot is released after the memory copy. 
% The shuffle data is stored in the reserved memory of SCache until all reduce tasks are pre-scheduled. 
% Then the shuffle data is pre-fetched according to the pre-scheduling results.  
% The application-context-aware memory management caches the shuffle data in memory before launching the reduce task.
% By applying these optimizations, SCache can help the DAG framework achieve a significant performance gain.  

% The main challenge to achieve this optimization is \textit{pre-scheduling reduce tasks without launching}. 
% Pre-scheduling without launching violates the design of most frameworks that launch a task after scheduling.
% Furthermore, randomly assigning reduce tasks using by na\"{i}ve scheduling schemes might result in a collision of two heavy tasks on one node.
% First, the complexity of DAG can amplify the defects of na\"{i}ve scheduling schemes. 
% In particular, randomly assigning reduce tasks might result in a collision of two heavy tasks on one node. 
% This collision can aggravate data skew, thus hurting the performance. 
% Second, pre-scheduling without launching violates the design of most frameworks that launch a task after scheduling.
% To address the above problems, we propose a heuristic task pre-scheduling scheme with shuffle data prediction and a task co-scheduler (Section \ref{opt}).
% To address the above problems, we propose a heuristic task pre-scheduling scheme (Section \ref{opt}).

% Another challenge is the \textit{in-memory data management}. 
% To prevent shuffle data touching the disk, SCache leverages extra memory to store the shuffle data. 
% To minimize the reserved memory while maximizing the performance gain, we propose two constraints: all-or-nothing and context-aware (Section \ref{impl}).

% {\color{black}
% We also propose a new performance model called \textit{Framework Resources Quantification} (FRQ) model.
% The FRQ model quantifies computing and I/O resources and visualizes the resources scheduling strategies of DAG frameworks in the time dimension.
% We use the FRQ model to assist in analyzing the deficiencies of resources scheduling and optimize it. 
% In the industrial production, the FRQ model can discover the irrationality of resource scheduling in big data analysis by revealing the relationships between the various phases (Section \ref{model}).
% }
% \begin{figure}
% 	\includegraphics[width=\linewidth]{fig/util}
% 	\caption{CPU Utilization and I/O Throughput of a Node During a Spark Single Shuffle Application}
% 	\label{fig:util}
% 	\vspace{-1em}
% \end{figure}

% {\color{black}
% % We implement SCache on both Spark and Hadoop MapReduce. 
% The performance of SCache is evaluated with both simulations and test-bed experiments on a 50-node Amazon EC2 cluster on both Apache Spark and Apache Hadoop.
% We evaluate SCache in three different benchmarks, including a standardized industry benchmark --- TPC-DS\footnote{http://www.tpc.org/tpcds/}. 
% % On Apache Spark, we conduct a basic test - \textit{GroupByTest}. We also evaluate the system with Terasort\footnote{https://github.com/ehiggs/spark-terasort} benchmark and standard workloads like TPC-DS\footnote{http://www.tpc.org/tpcds/} for multi-tenant modeling. On Apache Hadoop, we focus on Terasort benchmark.
% In a nutshell, SCache can eliminate explicit shuffle time by at most $89\%$ in varied applications. More impressively, SCache reduces $~40\%$ of overall completion time of TPC-DS queries on average on Apache Spark. On Apache Hadoop, SCache optimizes end-to-end Terasort completion time by $15\%$.
% }

\edited{\section{Background and Observations}\label{motivation}

In this section, we first study the typical shuffle characteristics (\ref{shuffle pattern}), and then spot the opportunities to achieve shuffle optimization (\ref{observation}).
\subsection{Characteristic of Shuffle} \label{shuffle pattern}

{\color{black}
In large scale data parallel computing, shuffle is designed to achieve an all-to-all data transfer among nodes.
The shuffle process can be further split into two parts: \textit{shuffle write} and \textit{shuffle read}. 
Shuffle write starts at the end of a map task and writes the partitioned map output data to local persistent storage. 
Shuffle read starts at the beginning of a reduce task and fetches the partitioned data from remote as its input.
Thus, Shuffle is I/O intensive, which might introduce a significant latency to the application. 
Reports show that $60\%$ of MapReduce jobs at Yahoo! and $20\%$ at Facebook are shuffle-heavy workloads \cite{shufflewatcher}. 
For those shuffle-heavy jobs, the shuffle latency may even dominate the completion time of jobs.
}
% For a clear illustration, we use \textit{map tasks} to define the tasks that produce shuffle data and use \textit{reduce tasks} to define the tasks that consume shuffle data.
% Note that one task may have both shuffle data generation and consumption in modern DAG framework. These tasks contain characteristic of both map task and reduce task. But these tasks won't change the behavior of shuffle. To avoid ambiguity, in the following paper, we will only use term of map task to represent those who produce shuffle output, and reduce task to represent those who consume shuffle output.
% \begin{figure}
% 	\centering
% 	\includegraphics[width=\linewidth]{fig/shuffle_process}
% 	\caption{Shuffle Overview}
% 	\label{fig:shuffle_process}
% \end{figure}

% \textbf{Overview of shuffle process}

% \subsubsection{Overview of shuffle process}
% Each map task partitions the result data (key, value pair) into several buckets according to the partition function (e.g., hash). 
% The total number of buckets equals the number of reduce tasks in the successive step.
% % shuffle data is produced by \textit{data partition}. For \textit{data partition}, 
% The shuffle process can be further split into two parts: \textit{shuffle write} and \textit{shuffle read}. 
% Shuffle write starts at the end of a map task and writes the partitioned map output data to local persistent storage. 
% Shuffle read starts at the beginning of a reduce task and fetches the partitioned data from remote as its input. 

%In short, shuffle is loosely coupled with application context and it's I/O intensive.
% \textbf{Impact of shuffle process}

% \subsubsection{Impact of shuffle process}
% {\color{black}
% Shuffle is I/O intensive, which might introduce a significant latency to the application. 
% Reports show that $60\%$ of MapReduce jobs at Yahoo! and $20\%$ at Facebook are shuffle-heavy workloads \cite{shufflewatcher}. 
% For those shuffle-heavy jobs, the shuffle latency may even dominate the completion time of jobs.
% For those shuffle-heavy jobs, the shuffle latency may even dominate Job Completion Time (JCT).
% For instance, a MapReduce trace analysis from Facebook shows that shuffle accounts for $33\%$ JCT on average, up to $70\%$ in shuffle-heavy jobs \cite{managing}.
% }

\subsection{Observations} \label{observation}
% Of course, shuffle is unavoidable in a DAG computing process. 
Can we mitigate or even remove the overhead of shuffle?
To analyze the design and implementation of shuffle, we run some Spark's \textit{GroupByTest} on a 5-node m4.xlarge EC2 cluster.
% To find the answers, we ran some typical Spark applications on a 5-node \texttt{m4.xlarge} EC2 cluster and analyzed the design and implementation of shuffle in some DAG frameworks.
Figure \ref{fig:util} shows the hardware utilization trace of one node of the experiment.
% Here we present the hardware utilization trace of one node running Spark's \textit{GroupByTest} in Figure \ref{fig:util} as an example. 
This job has 2 rounds of tasks for each node.
The \textit{Map Execution} is marked from the launch time of the first map task to the execution end time of the last one. 
The \textit{Shuffle Write} is marked from the beginning of the first shuffle write in the map stage. 
The \textit{Shuffle Read and Reduce Execution} is marked from the launch time of the first reduce task.
% Figure \ref{fig:util} reveals the performance information of two stages that are connected by shuffle. By analyzing the trace with Spark source code \cite{sparksource}, we propose the following observations.
% \begin{figure*}
% 	\includegraphics[width=\textwidth]{fig/util}
% 	\caption{CPU utiliazation and I/O throughput of a node during a Spark single shuffle application}
% 	\label{fig:util}
% \end{figure*}
\subsubsection{Coarse Granularity Resource Allocation}
{\color{black}
As shown in Figure \ref{fig:util}, the network transfer of shuffle data introduces an explicit I/O delay during \textit{shuffle read}.
Both \textit{shuffle write} and \textit{shuffle read} occupy the slot without significantly involving CPU.
The current coarse slot-task mapping results in an imbalance between task's resource demand and slot allocation thus decreasing the resource utilization. 

% {\color{red}
% When a slot is assigned to a task, it will not be released until the task completes (i.e., the end of shuffle write in Figure \ref{fig:util}). 
% On the reduce side, the network transfer of shuffle data introduces an explicit I/O delay during shuffle read (i.e., the beginning of shuffle read and execution in Figure \ref{fig:util}). 
% Meanwhile, both shuffle write and shuffle read occupy the slot without significantly involving CPU as presented in Figure \ref{fig:util}. 
% The current coarse slot-task mapping results in an imbalance between task's resource demand and slot allocation thus decreasing the resource utilization. 
% Unfortunately this defect exists not only in Spark but also Hadoop MapReduce and Tez. 
% A finer granularity resource allocation scheme should be provided to reduce these delays. 
% }

\subsubsection{Synchronized Shuffle Read}

The synchronized shuffle read requests cause several bursts of network traffic, which may result in network congestion and further slow down the network transfer.

% {\color{red}
% Almost all reduce tasks start shuffle read simultaneously. 
% The synchronized shuffle read requests cause a burst of network traffic. 
% As shown in Figure \ref{fig:util}, 
% the data transfer causes a high demand of network bandwidth, which may result in network congestion and further slow down the network transfer.
% It also happens in other frameworks that follow Bulk Synchronous Parallel (BSP) paradigm, such as Hadoop MapReduce, Dryad \cite{dryad}, etc.
% }

% The previous work \cite{coflow, managing} also proves that the network transfer can introduce significant overhead in DAG computing.

\subsubsection{Inefficient Persistent Storage Operation}
Both shuffle write and shuffle read are tightly coupled with task execution, which results in a blocking I/O operation.
This blocking I/O operation may introduce significant latency, especially in an I/O performance bounded cluster.
Besides, most of the DAG frameworks store shuffle data on disks (e.g., Spark, Hadoop MapReduce, Dryad, etc).
We argue that the memory capacity is large enough to store the short-living shuffle data during shuffle phases since several memory-based distributed storage systems have been proposed \cite{tachyon, ramcloud}.

% At first, both shuffle write and read are tightly coupled with task execution, which results in a blocking I/O operation. 
% This blocking I/O operation along with synchronized shuffle read may introduce significant latency, especially in an I/O performance bounded cluster.
% Besides, the legacy of storing shuffle data on disk is inefficient in modern clusters with large memory. 
% Compared to input dataset, the size of shuffle data is relatively small. 
% For example, the shuffle size of Spark Terasort is less than $25\%$ of input data. 
% The data reported in \cite{makingsense} also show that the amount of data shuffled is less than the input data by as much as $10\%-20\%$. 
% On the other hand, memory based distributed storage systems have been proposed \cite{tachyon, ramcloud} to move data back to memory, 
% but most of the DAG frameworks still store shuffle data on disks (e.g., Spark, Hadoop MapReduce, Dryad \cite{dryad}, etc).
% We argue that the memory capacity is large enough to store the short-living shuffle data with cautious management.

\subsubsection{Multi-round Tasks Execution}\label{multi}
Both experience and DAG framework manuals (e.g., Hadoop and Spark) recommend that multi-round execution of each stage will benefit the performance of applications.
Since the shuffle data becomes available when map tasks complete, 
and the network is idle during the map stage (\textit{Network TX} during map stage in Figure \ref{fig:util}), 
we propose an optimization that starts \textit{shuffle read} ahead of reduce stage to overlap the I/O operations in multi-round map tasks. 
% To achieve this optimization:
% \begin{itemize}
% 	\item Shuffle process should be decoupled from task execution to achieve a fine granularity scheduling scheme.
% 	\item Reduce tasks should be pre-scheduled without launching to achieve shuffle data pre-fetching.
% 	\item Shuffle process should be taken over and managed outside DAG frameworks to achieve a cross-framework optimization.
% \end{itemize}
% In the following section, we elaborate the methodologies to achieve three design goals.
}
}
\section{Shuffle Optimization}
\label{opt}
This section presents detailed methodologies to achieve shuffle optimization. 
{\color{black}
Firstly, we discuss the reason causes shuffle overhead in the DAG frameworks.
In the following subsection, we propose a shuffle data management system to decouple shuffle from execution. 
And we propose the pre-scheduling and pre-fetching to hide shuffle overhead in multi-round map tasks.
Furthermore, two heuristic algorithms (Algorithm \ref{hminheap}, \ref{mhminheap}) are used to improve the accuracy of prediction in the pre-scheduling.
}

% The out-of-framework shuffle data management is used to decouple shuffle from execution and provide a cross-framework optimization. 
% Two heuristic algorithms (Algorithm \ref{hminheap}, \ref{mhminheap}) and a co-scheduler is used to achieve shuffle data pre-fetching without launching tasks.


\begin{figure}
	\centering
	% \begin{subfigure}[b]{0.31\linewidth}
		% 	\includegraphics[width=\linewidth]{fig/hash_pre}
		% 	\caption{Linear Regression Prediction of Hash Partitioner\newline}
		% 	\label{fig:hash_pre}
	% \end{subfigure}
	\begin{subfigure}{0.75\linewidth}
		\includegraphics[width=\linewidth]{fig/range_pre_sample}
		\caption{Linear Regression and Sampling Prediction of Range Partitioner}
		\label{fig:range_pre_sample}
		% \vspace{1em}
	\end{subfigure}
	\begin{subfigure}{0.75\linewidth}
		\includegraphics[width=\linewidth]{fig/prediction_relative_error}
		\caption{Prediction Relative Error of Range Partitioner\newline}
		\label{fig:prediction_relative_error}
	\end{subfigure}
	\caption{Reduction Distribution Prediction}
	\label{fig:dis}
	% \vspace{-1em}
\end{figure}
\begin{figure}
	\centering
	\includegraphics[width=0.75\linewidth]{fig/sim} % second figure itself
	\caption{Stage Completion Time Improvement of OpenCloud Trace}
	\label{fig:sim}
\end{figure}

% {\color{black}
% \subsection{Observations}\label{observation}
% % DAG - shuffle - couple \& dependency - overhead in job - formula - how to optimize? - decouple \& pre-fetch - memory

% In distributed computing, we use shuffle to do all-to-all data transfer among the nodes.
% For a clear illustration, we divide a computing job into three phases: map, shuffle, and reduce.
% In most of the DAG frameworks, including Spark, Hadoop MapReduce, and Tez, shuffle phase is coupled with reduce phase which means that shuffle data transfer and reduce computing start simultaneously.
% On the reduce side, shuffle introduces an explicit overhead during shuffle read.
% % Because the reduce computing depends on the shuffle results, the waiting for the shuffle data causes the overhead.
% % because all reduce tasks start shuffle read simultaneously, 
% Furthermore, the synchronized shuffle read causes a burst of network traffic and further enlarges the overhead.

% % To better illustrate the shuffle optimization, we use this formula to represent the job completion time: \(T_{Job}=T_{Map}+T_{Reduce}+T_{overhead}\). \(T_{Map}\) represents the completion time of map phase and \(T_{Reduce}\) represents the reduce phase time without being affected by the coupled shuffle phase. 
% % We use \(T_{overhead}\) to represent the above-mentioned overhead caused by shuffle.
% % In the following subsections, we introduce how we decouple the shuffle and use pre-fetching and memory cache to mitigate the overhead.
% To mitigate the shuffle overhead, we propose an optimization to overlap the shuffle data transfer in multi-round map tasks, and use memory to cache the shuffle data. To achieve this optimization:
% \begin{itemize}
% 	\item Shuffle phase should be decoupled from reduce phase to achieve a better scheduling scheme.
% 	\item Reduce tasks should be pre-scheduled in map phase to achieve shuffle data pre-fetching.
% \end{itemize}
% % In Section \ref{model}, we also propose a performance model called the FRQ to model the optimization.
% }

\subsection{Decouple Shuffle from Execution}
{\color{black}
To decouple the shuffle from reduce tasks, we propose a shuffle data management system called SCache to take over all shuffle data from the DAG frameworks.
SCache provides two APIs named $putBlock$ and $getBlock$ to manage the shuffle data.
On the map side, after shuffle data blocks are produced, the map task uses $putBlock$ to transfer the data blocks to SCache.
%  instead of keeping them on local storage.
Inside the $putBlock$, SCache uses memory copy to move the shuffle data blocks to SCache's reserved memory and release the slot immediately.
After getting the shuffle data, SCache starts to shuffling data immediately.
% SCache gets the shuffle data as soon as the end of a map task's execution and starts to shuffling data immediately.
% SCache leverages the property of multi-round tasks to 
On the reduce side, due to the shuffle data pre-fetching of SCache, the reduce task use $getBlock$ to get the shuffle data from SCache.

SCache caches the shuffle data on the memory instead of storing on disk \edited{like} Spark or Hadoop MapReduce.
In this aspect, SCache optimizes the time of shuffle read and shuffle write since memory gets better I/O performance.
}
% To achieve the decoupling of map tasks and reduce tasks, the original shuffle write and read implementation in the current frameworks should be modified to apply the API of SCache.
% To prevent the release of a slot being blocked by shuffle write,  
% SCache provides a disk-write-like API named $putBlock$ to handle the storage of partitioned shuffle data blocks produced by a map task.
% Inside the $putBlock$, SCache uses memory copy to move the shuffle data blocks to SCache's reserved memory and release the slot immediately.
% Inside the $putBlock$, SCache uses memory copy to move the shuffle data blocks out of map tasks and store them in the reserved memory.
% After the memory copy, the slot will be released immediately.

% From the perspective of reduce task, SCache provides an API named $getBlock$ to replace the original implementation of shuffle read. 
% With the precondition of shuffle data pre-fetching, 
% the $getBlock$ leverages the memory copy to fetch the shuffle data from the local memory of SCache.

% \subsection{Pre-scheduling with Application Context}
{\color{black}
\subsection{Pre-scheduling and Pre-fetching Shuffle Data}
% multi-round -> how to optimize? -> pre-schedule -> pre-fetch -> formula
The pre-scheduling and pre-fetching are the most critical aspects of the optimization.
% Multi-round tasks execution is used to gain better parallelism in most of DAG frameworks. For example, Spark recommends 2-3 tasks per CPU core in tasks configuration.
% Since shuffle data is available as soon as the map task finished, and the network is idle in map phases, 
SCache starts pre-fetching and hides shuffle overhead into multi-round map tasks.
However, the shuffle data cannot be pre-fetched without the awareness of task-node mapping.

To optimize shuffle phase, we propose a co-scheduling scheme with two heuristic algorithms (Algorithm \ref{hminheap}, \ref{mhminheap}). 
The DAG frameworks follow the co-scheduler to start reduce tasks.
}
% The pre-scheduling and pre-fetching are the most critical aspects of the optimization. 
% The task-node mapping is not determined until tasks are scheduled by the scheduler of DAG framework. 
% Once the tasks are scheduled, the slots will be occupied to launch them. 
% And the shuffle data cannot be pre-fetched without the awareness of task-node mapping.
% We propose a co-scheduling scheme with two heuristic algorithms (Algorithm \ref{hminheap}, \ref{mhminheap}). 
% That is, the task-node mapping is established a priori, and then it is enforced by the co-scheduler when the DAG framework starts task scheduling. 

% \subsubsection{Random Mapping Problem}\label{randomassign}
% {\color{black}
% % Mapping tasks to nodes randomly and evenly is the simplest pre-scheduling way.
% To evaluate the performance of scheduling strategies, we use traces from OpenCloud\footnote{\label{fn:trace}http://ftp.pdl.cmu.edu/pub/datasets/hla/dataset.html} for the simulation.
% As shown in Figure \ref{fig:sim}, the baseline (i.e., red dotted line) is the stage completion time with Spark FIFO scheduling algorithm. 
% We run the simulation under three scheduling schemes: random mapping, Spark FIFO, and our heuristic MinHeap.
% The performance of random mapping drops as the round number grows. 
% This is because the data skew commonly exists in data-parallel computing \cite{skewtune}. 
% Several heavy tasks may be assigned to the same node, thus slowing down the whole stage. 
% In addition, randomly mapping also ignores the data locality between shuffle map output and reduce input, which may introduce extra network traffic in the cluster.
% }
% The simplest way of pre-scheduling is mapping tasks to nodes randomly and evenly. 
% We use traces from OpenCloud\footnote{\label{fn:trace}http://ftp.pdl.cmu.edu/pub/datasets/hla/dataset.html} for the simulation.
% In order to evaluate the effectiveness of random mapping, we use traces from OpenCloud\footnote{\label{fn:trace}http://ftp.pdl.cmu.edu/pub/datasets/hla/dataset.html} for the simulation.
% The average shuffle read time is $3.2\%$ of total reduce completion time.
% We remove the shuffle read time of each task and run the simulation under three scheduling schemes: random mapping, Spark FIFO, and our heuristic MinHeap.
% As shown in Figure \ref{fig:sim}, the baseline (i.e., red dotted line) is the stage completion time with Spark FIFO scheduling algorithm. 
% The performance of random mapping drops as the round number grows. 
% Random mapping works well when there is only one round of tasks, but the performance drops as the round number grows. 
% This is because that data skew commonly exists in data-parallel computing \cite{skewtune}. 
% Several heavy tasks may be assigned to the same node, thus slowing down the whole stage. 
% In addition, randomly assigned tasks also ignore the data locality between shuffle map output and reduce input, which may introduce extra network traffic in cluster.
	
\subsubsection{Shuffle Output Prediction}\label{shuffleprediction}
{\color{black}
% The problem of random mapping is obviously caused by application context (e.g., shuffle data size) ignorance. 
The simplest way of pre-scheduling is mapping tasks to nodes randomly and evenly.
But without considering job context, the random mapping scheduling shows poor performance due to data skew and ignoring data locality.
For the most DAG applications with random large scale input, 
the shuffle output can be predicted accurately by a linear regression model based on the observation that the ratio of map output size and input size are invariant given the same job configuration \cite{guo2017ishuffle}.
% Note that a balanced schedule decision can be made under the consideration of the size of each reduce task, and the size of a reduce task produced by one shuffle $reduceSize_i = \sum_{j=0}^{m} {BlockSize_{ji}}$, 
% where the $m$ is the number of map tasks that can be easily extracted from DAG information; 
% $BlockSize_{ji}$ represents the size of block which is produced by map $task_j$ for reduce $task_i$. 
% The final sizes of reduce tasks can be calculated by aggregating $reduceSize_i$ by reduce ID among all shuffle dependencies. 
% So the pre-scheduling can be made if the prediction of size of shuffle block is practical.
% For the most DAG applications with random large scale input, 
% the $BlockSize_{ji}$ in a particular shuffle can be predicted accurately by liner regression model (i.e., equation \ref{linearregresion}) based on observation that the ratio of map output size and input size are invariant given the same job configuration \cite{guo2017ishuffle}: 
% \begin{equation}
% \label{linearregresion}
% \begin{aligned}
% 	BlockSize_{ji} = a \times inputSize_j + b
% \end{aligned}
% \end{equation}
% The $inputSize_j$ is the input size of $j$th map task. 
% The $a$ and $b$ can be determined using the observed $inputSize_j$ and $BlockSize_{ji}$.
% Though the linear regression is stable in most scenarios, it can fail in some uncertainties introduced by sophisticated frameworks like Spark.  
However, the linear regression model can fail in some scenarios. 
For example, some customized partitions may cause large inconsistency between observed map output distribution and the final reduce input distribution. 
To illustrate the case, we present a particular spark job which uses the Spark RangePartitioner in Figure \ref{fig:range_pre_sample}. 
The observed map outputs are picked randomly. 
The job uses the Spark RangePartitioner and introduces an extremely high data skew.
Due to the extremely high data skew introduced by the partition, the linear regression model cannot fit the result well.
% The data partitioned by Spark RangePartitioner in Figure \ref{fig:range_pre_sample} results in a deviation from the linear regression model, because the RangePartitioner might introduce an extreme high data locality skew. 
% We present two particular examples with 20 tasks respectively in Figure \ref{fig:hash_pre} and Figure \ref{fig:range_pre_sample}. 
% The data in are normalized to $0-1$ because the prediction of SCache only produces the data distribution instead of the real size. 
% In Figure \ref{fig:range_pre_sample}, we use the distribution of second shuffle of Spark Terasort \cite{spark-tera} that are partitioned by Spark RangePartitioner \cite{apachespark}. 
% In Figure \ref{fig:hash_pre} and Figure \ref{fig:range_pre_sample}, we use different datasets with different partitioners, and normalize the distribution to $0-1$ to fit in one figure. 
% The observed map outputs are randomly picked. 
% With a random input and a hash partitioner in Figure \ref{fig:hash_pre}, the distribution of observed map output is close to the final reduce input distribution. 
% The prediction results also fit them well. 
% However, the data partitioned by Spark RangePartitioner in Figure \ref{fig:range_pre_sample} results in a deviation from the linear regression model, because the RangePartitioner might introduce an extreme high data locality skew. 
That is, for one reduce task, almost all of the input data are produced by a particular map task (e.g., the observed map tasks only produce data for reduce task 0-5 in Figure \ref{fig:range_pre_sample}).
The data locality skew results in a missing of other reduce tasks' data in the observed map outputs.

To solve this problem, we propose a new methodology named \textit{weighted reservoir sampling}.
SCache uses this method instead of the linear regression to predict output when using a RangePartitioner or a customized non-hash partitioner.
% we introduce another methodology, named \textit{weighted reservoir sampling}, as a substitution of linear regression. 
% Note that linear regression will be replaced only when a RangePartitioner or a customized non-hash partitioner occurs. 
For each map task, we use classic reservoir sampling to randomly pick $s \times p$ of samples, where $p$ is the number of reduce tasks and $s$ is a tunable number. 
After that, the map function is called locally to process the sampled data. 
Finally, the partitioned outputs are collected with the $InputSize_j$ as the weight of the samples.
The $inputSize_j$ is the input size of $j$th map task. 
% Note that sampling does not consume the input data of map tasks. 
% The $BlockSize_{ji}$ can be calculated by:
$BlockSize_{ji}$ represents the size of block which is produced by map $task_j$ for reduce $task_i$:
\begin{equation}
\label{equationsample}
\begin{aligned}
	BlockSize_{ji} &= {{InputSize_j \times \frac{sample_i}{s \times p}}} \\
	sample_i &= \text{number of samples for $reduce_i$}
\end{aligned}
\end{equation}
In Figure \ref{fig:range_pre_sample}, when $s$ is set to $3$, the result of sampling prediction is much better than linear regression. 
Figure \ref{fig:prediction_relative_error} further proves that the sampling prediction can provide a much more accurate result than the linear regression. 
% The classic reservoir sampling is designed for randomly choosing \textit{k} samples from \textit{n} items, where \textit{n} is either a very large or an unknown number \cite{reservoir}. 
% In Figure \ref{fig:range_pre_sample}, when $s$ is set to $3$, the result of sampling prediction is much better than linear regression. 
% The variance of the normalization between sampling prediction and reduce distribution is because the standard deviation of the prediction results is relatively small compared to the average prediction size, which is $0.0015$ in this example. 
% Figure \ref{fig:prediction_relative_error} further proves that the sampling prediction can provide precise result even in the dimension of absolute input size of reduce task. 
% On the other hand, the result of linear regression comes out with a large relative error. 
% Though the weighted reservoir sampling is precise, it also introduces extra overhead. 
% We will show the overhead evaluation of sampling in Section \ref{evaluation}.
% {\color{black}

% Figure \ref{fig:prediction_relative_error} proves that the sampling prediction can provide much more accurate result than the linear regression. 
% We will show the overhead evaluation of sampling in Section \ref{evaluation}. 
% }

}
During both predictions, the composition of each reduce partition is calculated as well. We define $prob_i$ as
\begin{equation}
\label{equationprob}
\begin{aligned}
	prob_i &= \max_{0 \leq j \leq m} \frac{BlockSize_{ji}}{reduceSize_i} \\
	m &= \text{number of map tasks}
\end{aligned}
\end{equation}
% Note that a balanced schedule decision can be made under the consideration of the size of each reduce task, and the size of a reduce task produced by one shuffle $reduceSize_i = \sum_{j=0}^{m} {BlockSize_{ji}}$, 
% where the $m$ is the number of map tasks that can be easily extracted from DAG information;
The $reduceSize_i$ represents the size of a reduce task, which is represented by $reduceSize_i = \sum_{j=0}^{m} {BlockSize_{ji}}$.
We use $prob_i$ to achieve a better data locality while performing shuffle pre-scheduling. 
% This parameter is used to achieve a better data locality while performing shuffle pre-scheduling. 

\begin{minipage}{0.95\columnwidth}
\begin{algorithm}[H]
\caption{Heuristic MinHeap Scheduling for Single Shuffle}
\label{hminheap}
	\begin{algorithmic}[1]
	\small
	\Procedure{schedule}{$m, host\_ids, p\_reduces$}
		\State $m\gets$ partition number of map tasks
		\State $R\gets$ sort $p\_reduces$ by size in \edited{decreasing} order
		\State $M\gets$ min-heap $\left\{ host\_id \rightarrow \left( \left[ reduces \right], size \right) \right\}$
		\State $idx\gets 0$
		\While{$idx <$ len$R$}
		% \Comment{Schedule reduces by MinHeap}
		\State $M\left[0\right].size \mathrel{+}= R\left[idx\right].size$
		\State $M\left[0\right].reduces.append\left(R\left[idx\right]\right)$
		\State $R\left[idx\right].assigned\_id \gets M \left[0\right].host\_id$
		\State Sift down $M\left[0\right]$ by $size$
		\State $idx\gets idx+1$
		\EndWhile
		\State $max\gets$ maximum size in $M$
		% \State Convert $M$ to mapping $\left\{ host\_id \rightarrow \left( \left[ rid\_arr \right], size \right) \right\}$
		\ForAll{$reduce$ in $R$}
		\Comment{Heuristic locality swap}
			\If{$reduce.assigned\_id \neq reduce.host\_id$}
				\State $p\gets reduce.prob$
				\State $norm\gets \left(p-1/m\right)/\left(1-1/m\right)/10$
				\State $upper\_bound \gets \left(1 + norm\right) \times max$
				\State SWAP\_TASKS$\left(M, reduce, upper\_bound\right)$
			\EndIf
		\EndFor\newline
		\Return $M$
	\EndProcedure
	\Procedure{swap\_tasks}{$M, reduce, upper\_bound$}
		\State Swap tasks between node $host\_id$ and node $assigned\_id$
		\State of $reduce$ without exceeding the $upper\_bound$
		\State of both nodes.\newline
		% \State Return if it is impossible.
		\Return
	\EndProcedure
	\end{algorithmic}
\end{algorithm}
\end{minipage}

\subsubsection{Heuristic MinHeap Scheduling}\label{h-minheap}
% As long as the input sizes of reduce tasks are available, the pre-scheduling is a classic scheduling problem without considering the data locality. 
% But ignoring the data locality can introduce extra network transfer. 
% In order to balance load while minimizing the network traffic, we present the Heuristic MinHeap Scheduling algorithm (Algorithm \ref{hminheap}).
To balance load while minimizing the network traffic, we present the \textit{Heuristic MinHeap Scheduling} algorithm (Algorithm \ref{hminheap}).
{\color{black}
% For the pre-scheduling itself (i.e., the first $while$ in Algorithm \ref{hminheap}), the algorithm maintains a min-heap to simulate the load of each node and applies the longest processing time rule (LPT)\footnote{http://www.designofapproxalgs.com/} to achieve $4/3\text{-}approximation$ optimum. 
To keep the pre-scheduling load balance, we maintain a min-heap in the first $while$ loop (i.e., line 6-11).
We use the min-heap to simulate the load of each node and apply the longest processing time rule (LPT)\footnote{http://www.designofapproxalgs.com/} to achieve $4/3\text{-}approximation$ optimum.
Spark FIFO only can achieve $2\text{-}approximation$ optimum.
}
% Since the sizes of tasks are considered while scheduling, \textit{Heuristic MinHeap Scheduling} can achieve a shorter makespan than Spark FIFO which is a $2\text{-}approximation$ optimum. 
% Simulation of OpenCloud trace in Figure \ref{fig:sim} also shows that \textit{Heuristic MinHeap Scheduling} has a better improvement (average 5.7\%) than the Spark FIFO (average 2.7\%).
After pre-scheduling, the task-node mapping will be adjusted according to the locality. 
The $SWAP\_TASKS$ will be triggered when the $host\_id$ of a task does not equal the $assigned\_id$.
Based on the $prob$, the normalized probability $norm$ is calculated as a bound of performance degradation. 
Inside the $SWAP\_TASKS$, tasks will be selected and swapped without exceeding the $upper\_bound$.
Since the algorithm only maintains a min-heap and traverses $reduce$ for swapping, the algorithm needs $O(n)$ operations. 

{\color{black}
To evaluate \textit{Heuristic MinHeap Scheduling} algorithm, we use traces from OpenCloud\footnote{\label{fn:trace}http://ftp.pdl.cmu.edu/pub/datasets/hla/dataset.html} for the simulation.
% As shown in Figure \ref{fig:sim}, the baseline (i.e., red dotted line) is the stage completion time with Spark FIFO scheduling algorithm. 
As shown in Figure \ref{fig:sim}, we run the simulation under three scheduling schemes: Random Mapping, Spark FIFO, and heuristic MinHeap.
After balancing load based on the job context (e.g. shuffle size, data locality), \textit{Heuristic MinHeap Scheduling} has a better improvement (average 5.7\%) than Spark (average 2.7\%) and Random Mapping.
%  (becomes negative after the round one).
}

\begin{minipage}{0.95\columnwidth}
	\begin{algorithm}[H]
	\caption{Accumulated Heuristic Scheduling for Multi-Shuffles}
	\label{mhminheap}
		\begin{algorithmic}[1]
		\small
		\Procedure{m\_schedule}{$m, host\_id, p\_reduces, \text{\textit{shuffles}}$}
			\State $m\gets$ partition number of map tasks
			\Comment \textit{shuffles} are the previous schedule result 
			\ForAll{$r$ in $p\_reduces$}
				\State $r.size \mathrel{+}= \text{\textit{shuffles}}\left[r.rid\right].size$
				\State $new\_prob\gets \text{\textit{shuffles}}\left[r.rid\right].size / r.size$
				\If{$new\_prob\geq r.prob$}
					\State $r.prob\gets new\_prob$
					\State $r.host\_id\gets \text{\textit{shuffles}}\left[r.rid\right].assigned\_host$
				\EndIf
			\EndFor
			\State $M\gets$ $SCHEDULE\left(m, host\_id, p\_reduces\right)$
			\ForAll{$host\_id$ in $M$}
				\Comment Re-shuffle
				\ForAll{$r$ in $M\left[host\_id\right].reduces$}
					\If{$host\neq \text{\textit{shuffles}}\left[r.rid\right].assigned\_host$}
					\State Re-shuffle data to $host$
					\State $\text{\textit{shuffles}}\left[r.rid\right].assigned\_host\gets host$
					\EndIf
				\EndFor
			\EndFor
			\Return $M$
		\EndProcedure
		\end{algorithmic}
	\end{algorithm}
\end{minipage}

\subsubsection{Cope with Multiple Shuffle Dependencies}
{\color{black}
A reduce stage can have more than one shuffle dependency in the current DAG computing frameworks.
To cope with multiple shuffle dependencies, we present the \textit{Accumulated Heuristic Scheduling} algorithm.
}
% The technique mentioned in Section \ref{shuffleprediction} can only handle an ongoing shuffle. 
% For those pending shuffles, it is impossible to predict their sizes. 
% This problem can be solved by having all map tasks of pending shuffles launched simultaneously. 
% But doing this introduces large overhead such as extra task serialization. 
% To avoid violating the optimization from framework, we present the \textit{Accumulated Heuristic Scheduling} algorithm to cope with multiple shuffle dependencies.
As illustrated in Algorithm \ref{mhminheap}, the sizes of previous \textit{shuffles} scheduled by \textit{Heuristic MinHeap Scheduling} are counted. 
When a new shuffle starts, the predicted $size$, $prob$, and $host\_id$ in $p\_reduces$ are accumulated with previous \textit{shuffles}. 
After scheduling, if the new $assigned\_id$ of a reduce task is not equal the original one, a re-shuffle will be triggered to transfer data to the new host. 
This re-shuffle is rare since the previous shuffle data contributes a huge composition (i.e., high $prob$) after the accumulation, 
which leads to a higher probability of tasks swap in $SWAP\_TASKS$. 

To traverse $reduce$ for accumulating previous $shuffle$ and re-shuffling data, the algorithm needs $O(n)$ operations.

\section{Implementation}\label{impl}
This section presents an overview of the implementation of SCache. 
% {\color{red}
% Here we use Spark as an example of DAG framework to illustrate the work flow of shuffle optimization. 
% We will first present the system overview in Subsection \ref{arch} while the following two subsections focus on the two constraints on memory management.
% }
{\color{blue}
We first present the system overview and the detail of sampling in Subsection \ref{arch}. 
The following \ref{memorymanage} subsection focuses on two constraints on memory management.
In Subsection \ref{crossframework}, we evaluate the cross-framework capability of SCache. 
At last, we discuss the cost of adapting SCache and fault tolerance. 
}


\subsection{System Overview}\label{arch}
SCache consists of three components: a distributed shuffle data management system, a DAG co-scheduler, and a %{\color{red}daemon inside Spark.}
{\color{blue} worker daemon. As a plug-in system, SCache needs to rely on a DAG framework.} As shown in Figure \ref{fig:architecture}, SCache employs the legacy master-slaves architecture like GFS \cite{gfs} for the shuffle data management system. 
The master node of SCache coordinates the shuffle blocks globally with application context. The worker node reserves memory to store blocks.
The coordination provides two guarantees: (a) data is stored in memory before tasks start and (b) data is scheduled on-off memory with all-or-nothing and context-aware constraints. 
The daemon bridges the communication between %{\color{red}Spark}
{\color{blue} DAG framework} and SCache. The co-scheduler is dedicated to pre-schedule reduce tasks with DAG information and enforce the scheduling results to %{\color{red}Spark scheduler}
 {\color{blue} original scheduler in framework}.

% SCache consists of three components: a distributed shuffle data management system, a DAG co-scheduler, and a {\color{blue} worker daemon}. As shown in Figure \ref{fig:arch}, SCache employs the legacy master-slaves architecture like GFS \cite{gfs} for shuffle data management system. 

% The master node of SCache coordinates the shuffle blocks globally with application context. The worker node reserves memory to store blocks.
% The coordination provides two guarantees: (a)data is stored in memory before tasks start and (b)data is scheduled on-off memory with all-or-nothing and context-aware constraints. 
% The daemon bridges the communication between {\color{blue}DAG framework} and SCache. The co-scheduler is dedicated to pre-schedule reduce tasks with DAG information and enforce the scheduling results to {\color{blue} original scheduler in framework master}.

% {\color{red}
% When a Spark job starts, the DAG will be first generated. 
% During the DAG generation, the shuffle dependencies among Resilient Distributed Datasets (RDDs) will then be submitted through the daemon process in Spark driver. 
% }
{\color{blue}
When a DAG job is submitted, the DAG information will be generated in framework task scheduler. 
Before the computing tasks begin, the shuffle dependencies are determined based on DAG.
}
For each shuffle dependency, the shuffle ID, the type of partitioner, the number of map tasks, and the number of reduce tasks are included.  If there is a specialized partitioner, such as range partitioner, in the shuffle dependencies, the daemon will insert a sampling application before the {\color{blue}computing job}. We will elaborate the sampling procedure in the Section \ref{sampling}.
% \begin{minipage}{\columnwidth}
% \begin{algorithm}[H]
% \caption{Heuristic MinHeap Scheduling for Single Shuffle}
% \label{hminheap}
% 	\begin{algorithmic}[1]
% 	\small
% 	\Procedure{\Large schedule}{$m, host\_ids, p\_reduces$}
% 		\State $m\gets$ partition number of map tasks
% 		\State $R\gets$ sort $p\_reduces$ by size in descending order
% 		\State $M\gets$ min-heap $\left\{ host\_id \rightarrow \left( \left[ reduces \right], size \right) \right\}$
% 		\State $idx\gets$ len$\left(R\right) - 1$
% 		\While{$idx \geq 0$}
% 		\Comment{Schedule reduces by MinHeap}
% 		\State $M\left[0\right].size \mathrel{+}= R\left[idx\right].size$
% 		\State $M\left[0\right].reduces.append\left(R\left[idx\right]\right)$
% 		\State $R\left[idx\right].assigned\_host \gets M \left[0\right].host\_id$
% 		\State Sift down $M\left[0\right]$ by $size$
% 		\State $idx\gets idx-1$
% 		\EndWhile
% 		\State $max\gets$ maximum size in $M$
% 		\State Convert $M$ to mapping $\left\{ host\_id \rightarrow \left( \left[ rid\_arr \right], size \right) \right\}$
% 		\ForAll{$reduce$ in $R$}
% 		\Comment{Heuristic swap by locality}
% 			\If{$reduce.assigned\_id \neq reduce.host\_id$}
% 				\State $p\gets reduce.prob$
% 				\State $norm\gets \left(p-1/m\right)/\left(1-1/m\right)/10$
% 				\State $upper\_bound \gets \left(1 + norm\right) \times max$
% 				\State SWAP\_TASKS$\left(M, reduce, upper\_bound\right)$
% 			\EndIf
% 		\EndFor
% 		% \Comment{$m$ is the number of input data}
% 		% \Comment{$r$ is partition number of reduces}
% 		% \Comment{$hosts$ is array of (hostid, partitionids[], size)}
% 		% \Comment{$c$ is $r*m$ array of composition data}
% 		% \Comment{$pSize$ is $r$ size array of predicted size of reduces}
% 		\Return $M$
% 	\EndProcedure
% 	\Procedure{\Large swap\_tasks}{$M, reduce, upper\_bound$}
% 		\State $reduces \gets M\left[reduce.host\_id\right].reduce$	
% 		\State $candidates \gets$ Select from $reduces$ that $assigned\_id \neq host\_id$ \textbf{and} total size closest to $reduce.size$
% 		\State $\Delta size \gets sizeOf\left(candidates\right) - reduce.size$
% 		\State $size\_host \gets M\left[reduce.host\_id\right].size - \Delta size$
% 		\State $size\_assigned \gets M\left[reduce.assigned\_id\right].size + \Delta size$
% 		\If{$size\_host\leq upper\_bound$ \textbf{and} \\
% 			\qquad \; $size\_assigned\leq upper\_bound$}
% 			\State Swap $candidates$ and $reduce$
% 			\State Update $size$ in $M$
% 			\State Update $assigned\_host$ in $candidates$ and $reduce$
% 		\EndIf
% 	\EndProcedure
% 	\end{algorithmic}
% \end{algorithm}
% \end{minipage}

When a map task finishes computing, the shuffle write implementation of DAG framework is modified to call the SCache API and move all the blocks out of {\color{blue}framework worker} through memory copy. 
After that, the slot will be released (without being blocked on disk operations).
When a block of the map output (i.e., "map output" in Figure \ref{fig:shuffle}) is received, the SCache worker will send the block ID and the size to the master.
If the collected map output data reach the observation threshold, the DAG co-scheduler will run the scheduling Algorithm \ref{hminheap} or \ref{mhminheap} to pre-schedule the reduce tasks and then broadcast the scheduling result to start pre-fetching on each worker.
% After that, when a map task is finished, each node will receive a broadcast message. 
SCache worker will filter the reduce tasks' IDs that are scheduled on itself and start pre-fetching shuffle data from the remote. 
% {\color{red}
% To enforce SCache pre-scheduled tasks -- node mapping, we insert some lines of codes in Spark DAG Scheduler.
% For RDDs with shuffle dependencies, Spark DAG scheduler will consult SCache master to get the preferred location for each partition and set \textit{NODE\_LOCAL} locality level on corresponding reduce tasks.
% }
{\color{blue}
In order to force DAG framework to run according to the SCache pre-scheduled results, we insert some lines of codes in framework scheduler.
After modification, DAG scheduler consults SCache co-scheduler to get the preferred location for each task.
}
% In the case of extreme skew scenario, such as Figure \ref{fig:range_pre_sample}, Heuristic MinHeap trades about 0.05\% percent of stage completion time for 99\% reduction of shuffle data transmission through network by heuristicly swapping tasks.
% \begin{minipage}{\columnwidth}
% \begin{algorithm}[H]
% \caption{Accumulate Heuristic Scheduling for Multi-Shuffles}
% \label{mhminheap}
% 	\begin{algorithmic}[1]
% 	\small
% 	\Procedure{\Large m\_schedule}{$m, host\_id, p\_reduces, shuffles$}
% 		\State $m\gets$ partition number of map tasks
% 		\Comment $shuffles$ is the previous schedule result 
% 		\ForAll{$r$ in $p\_reduces$}
% 			\State $r.size \mathrel{+}= shuffles\left[r.rid\right].size$
% 			\If{$shuffles\left[r.rid\right].size\geq r.size \times r.prob$}
% 				\State $r.prob\gets shuffles\left[r.rid\right].size / r.size$
% 				\State $r.host\_id\gets shuffles\left[r.rid\right].assigned\_host$
% 			\EndIf
% 		\EndFor
% 		\State $M\gets$ $SCHEDULE\left(m, host\_id, p\_reduces\right)$
% 		\ForAll{$host\_id$ in $M$}
% 			\Comment Re-shuffle
% 			\ForAll{$r$ in $M\left[host\_id\right].reduces$}
% 				\If{$host\neq shuffles\left[r.rid\right].assigned\_host$}
% 				\State Re-shuffle data to $host$
% 				\State $shuffles\left[r.rid\right].assigned\_host\gets host$
% 				\EndIf
% 			\EndFor
% 		\EndFor
% 		\Return $M$
% 	\EndProcedure
% 	\end{algorithmic}
% \end{algorithm}
% \end{minipage}

\subsubsection{Reservoir Sampling}\label{sampling}
If the submitted shuffle dependencies contained a RangePartitioner or a customized non-hash partitioner, {\color{blue}the SCache master will send a sampling request to the framework master}. 
The sampling job uses a reservoir sampling algorithm \cite{reservoir} on each partition. 
For the sample number, it can be tuned to balance the overhead and accuracy. 
The sampling job randomly selects some items and performs a local shuffle with partitioner (see Figure \ref{fig:sample}). 
At the same time, the items number is counted as the weight. 
These sampling data will be aggregated by reduce task ID on SCache master to predict the reduce partition size. 
After the prediction, SCache master will call Algorithm \ref{hminheap} or \ref{mhminheap} to do the pre-scheduling.

% \begin{figure}
% 	\centering
% 	\includegraphics[width=0.8\linewidth]{fig/arch}
% 	\caption{\color{red}SCache Architecture}
% 	\label{fig:arch}
% 	\vspace{-0.5em}
% \end{figure}

\begin{figure}
	\centering
	\begin{minipage}{.47\textwidth}
		\centering
		\includegraphics[width=\linewidth]{fig/architecture}
		\caption{\color{blue}SCache Architecture}
		\label{fig:architecture}
		% \vspace{-0.5em}
	\end{minipage}\hfill
	\begin{minipage}{.42\textwidth}
		\centering
		\includegraphics[width=\linewidth]{fig/sample}
		\caption{Reservoir Sampling of One Partition}
		\label{fig:sample}
		% \vspace{-1em}
	\end{minipage}
\end{figure}

\subsection{Memory Management}\label{memorymanage}
As mentioned in Section \ref{observation}, though the shuffle size is relatively small, memory management should still be cautious enough to limit the effect of performance of DAG framework.
When the size of cached blocks reaches the limit of reserved memory, SCache flushes some of them to the disk temporarily, and re-fetches them when some cached shuffle blocks are consumed or pre-fetched. 
To achieve the maximum overall improvement, SCache leverages two constraints to manage the in-memory data-all-or-nothing and context-aware-priority.



\subsubsection{All-or-Nothing Constraint}
This acceleration of in-memory cache of a single task is necessary but insufficient for a shorter stage completion time. 
Based on the observation in Section \ref{multi}, in most cases one single stage contains multi-rounds of tasks. 
If one task missed a memory cache and exceeded the original bottleneck of this round, that task might become the new bottleneck and then slow down the whole stage. 
PACMan \cite{pacman} has also proved that for multi-round stage/job, the completion time improves in steps when $n\times number\ of\ tasks\ in\ one\ round$ of tasks have data cached simultaneously. 
Therefore, the cached shuffle blocks need to match the demand of all tasks in one running round at least. We refer to this as the all-or-nothing constraint.

According to all-or-nothing constraint, SCache master leverages the pre-scheduled results to determine the bound of each round, and sets blocks of one round as the minimum unit of storage management.
For those incomplete units, SCache marks them as the lowest priority.
% Following the all-or-noting constraint can maximum the improvement in stage completion time by using reserved memory efficiently.

\subsubsection{Context-Aware-Priority Constraint}
Unlike the traditional cache replacement schemes such as MIN \cite{min}, the cached shuffle data will only be used once (without failure), but the legacy cache managements are designed to improve the hit rate.
SCache leverages application context to select victim storage units when the reserved memory is full.

At first, SCache flushes blocks of the incomplete units to disk cluster-widely.
If all the units are completed, SCache selects victims based on two factors-\textit{inter-shuffle} and \textit{intra-shuffle}.
\begin{itemize}[noitemsep]
	\item Inter-shuffle: SCache master follows the scheduling scheme of Spark to determine the inter-shuffle priority. 
	For example, Spark FIFO scheduler schedules the tasks of different stages according to the submission order. 
	So SCache sets the priorities according to the submission time of each shuffle.
	% For a FAIR scheduler, Spark balances the resource among task sets, which leads to a higher priority for those having more tasks unfinished. 
	% So SCache sets priorities from high to low in a descending order of remaining storage units of a shuffle. 
	% For a FIFO scheduler, Spark schedules the task set that is submitted first. 
	% So SCache sets the priorities according to the submit time of each shuffle unit.
	\item Intra-shuffle: The intra-shuffle priorities are determined according to the task scheduling inside a stage.
	For example Spark schedules tasks with smaller ID at first. 
	Based on this, SCache can assign the lower priority to storage units with a larger task ID.
\end{itemize}

% {\color{red}
% \subsection{Cost of adapting DAG frameworks}
% SCache provides API through RPC, such as \textit{putBlock(blockId)}, \textit{getBlock(blockId)}, and \textit{getScheduleResult(shuffleId)}. The concise design makes it easy to adapt DAG frameworks to enable SCache optimization. For example, it only takes about 500 lines of code in Spark to integrate SCache. By a glance of Hadoop source code, we believe that the costs of enabling SCache on MapReduce \cite{hadoop} and YARN \cite{yarn} based DAG computing framework, such as Tez \cite{tez}, are also very low.
% }
{\color{blue}
\subsection{Analysis of cross-framework capability}\label{crossframework}
Shuffle optimization of SCache inevitably requires the modification on DAG frameworks. SCache provides APIs through RPC, such as \textit{putBlock (blockId)}, \textit{getBlock (blockId)}, and \textit{getScheduleResult (shuffleId)}. In order to use SCache, we mainly need to modify two parts of the frameworks: (a) The DAG scheduler should provide the DAG information and follow the pre-scheduled result of SCache; (b) The shuffle data should be transferred to SCache Storage Management.

To prove the cross-framework capability of SCache, we adapt SCache on Hadoop MapReduce and Spark respectively. In Hadoop MapReduce, we modify codes in ResourceManager, MapTask, and ReduceTask to call SCache APIs through RPC. It takes about 380 lines of code. In Spark, we mainly modify DAGScheduler and the corresponding data fetcher. 
It only takes about 500 lines of code. Such hundreds of lines of code modification are very small compared to the hundreds of thousands of lines of code in DAG framework. We believe that the costs of enabling SCache on other DAG computing frameworks, such as Tez \cite{tez}, are also very low.
}

\subsection{Fault tolerance}\label{fault}
Due to the characteristic of shuffle data (e.g., short-lived, write-once, read-once), we believe fault tolerance is not a crucial goal of SCache at present. 
We plan to implement SCache master with Apache ZooKeeper \cite{zookeeper} to provide constantly service. 
If a failure happened inside the SCache worker, SCache daemon could block the shuffle write/read operations until the worker process restarted without violating the correctness of DAG computing.
A possible way to handle this failure is selecting some backup nodes to store replications. 
But the replications can introduce a significant network overhead \cite{availability}.  
Currently, we left the sever faults (e.g., the failure of a node) to the DAG frameworks. 
We believe it is a more promising way because most DAG frameworks have more advanced fast recovery schemes on the application layer, such as paralleled recovery of Spark. 
Meanwhile, SCache can still provide shuffle optimization during the recovery.





% \subsubsection{Fault Tolerance}
% To prevent the machine failure in cluster leading to inconsistency SCache, the master node will log the meta data of shuffle register and scheduling on the disk. Since we remove the shuffle transfer from the critical path of DAG computing, the disk log will not introduce extra overhead to the DAG framworks. Note that the master can be implemented with Apache ZooKeeper \cite{zookeeper} to provide constantly service to DAG framework.
% At the same time, every work node will send a heartbeat to master to report status. If a failure of work node is detected, the master will the do a simple re-schedule. For those scheduled shuffle units, the master assgins the tasks to other workers with more lightweight workload evenly. Then the new assigned worker will fetch the data again. For the incomplete in memory map blocks on the failure node, SCache simply ignore them since DAG framework will schedule the failure map tasks on another node.

{\color{blue}
\section{Framework Resources Quantification Model}\label{model}

\begin{figure*}
    \centering
	\includegraphics[width=0.85\textwidth]{fig/model_basic}
	\caption{\color{blue}FRQ Model}
    \label{fig:model_basic}
    \vspace{-1em}
\end{figure*}

In this chapter, we introduce \textit{Framework Resources Quantification}(FRQ) model to describe the performance of DAG frameworks.
FRQ model quantifies computing and I/O resources and displays them in time dimension. According to FRQ model, we can calculate the execution time required by the application under any circumstances, including different DAG framework, hardware environment, and so on. Therefore FRQ model is able to help us analyze the resources scheduling of DAG framework and evaluate their performance. We will first introduce FRQ model in Subsection \ref{model_overview}. In the following Subsection \ref{model_analysis}, we will use FRQ model to describe three different computation job and analyze their performance. In the last Subsection \ref{model_verification}, we will use the actual experimental results to verify the FRQ model.

\subsection{The FRQ Model}\label{model_overview}
The current distributed parallel computing frameworks mostly use DAG(Directed acyclic graph) to describe computation logic. A shuffle phase is required between each adjacent DAG computation phase. In order to better analyze the relationship between the computation phase and the shuffle phase, we propose FRQ model. After quantifying computing and I/O resources, FRQ model is able to describe different resource scheduling strategies. For convenience, we introduce FRQ model by taking a simple MapReduce as an example in this section.

Figure \ref{fig:model_basic} shows how the FRQ model describes a MapReduce task. FRQ model has five input parameters:
\begin{itemize}
	\item Input Data Size\((D)\): The data size of the computation phase.
	\item Data Conversion Rate\((R)\): The conversion rate of the input data to the shuffle data during a computation phase. This conversion rate depends on the algorithm used in the computation phase.
    \item Computation Round Number\((N)\): The number of rounds needed to complete the computation phase using the current computation resources. The number of rounds depends on the current computation resources and the settings of the computation job. Take Hadoop MapReduce as an example, suppose we have 50 CPUs and enough memory, the Map phase consists of 200 map tasks. Then we need 4 rounds of computation to complete the Map phase.
    \item Computation Speed\((V_{i})\): 
    The computation speed for each computation phase. The computation speed depends on the algorithm used in the computation phase.
    \item Shuffle Speed\((V_{Shuffle})\): 
    Transmission speed during shuffle. Shuffle speed depends on Network and storage device bandwidth.
\end{itemize}

We can calculate the execution time of each phase of the job with these five parameters. Obviously, the total execution time of job in this case is the sum of the Map phase time and Reduce phase time:
\begin{equation}
\label{equation_Tjob}
\begin{aligned}
    T_{Job} &= T_{Map} + T_{Reduce}
\end{aligned}
\end{equation}
Map phase time depends on input data size and Map computation speed:
\begin{equation}
\label{equation_Tmap}
\begin{aligned}
    T_{Map} &= {{\frac{D}{V_{Map}}}}
\end{aligned}
\end{equation}
The Reduce phase time formula is as follows:
\begin{equation}
\label{equation_Treduce}
\begin{aligned}
    T_{Reduce} &= \frac{D \times R}{V_{Reduce}} + K \times T_{P\_Shuffle}
\end{aligned}
\end{equation}
\(\frac{D \times R}{V_{Reduce}}\) represents the ideal computation time, and \(K \times T_{P\_Shuffle}\) represents the calculated overhead. \(K\) is empirical value.The overhead depends on the parallel time of shuffle phase and the Reduce phase.The parallel time is denoted by \(T_{P\_Shuffle}\). And the total time of shuffle phase is represented by \(T_{Shuffle}\). Because the computatinon of the Reduce phase relies on the data transfer results of the Shuffle phase, a portion of the computation in the Reduce phase need to wait for the transfer results. The overhead is caused by these waiting. The FRQ model uses K to indicate the extent of the waiting.

\begin{equation}
\label{equation_Tshuffle}
\begin{aligned}
    T_{Shuffle} &= {{\frac{D}{V_{Shuffle}}}}
\end{aligned}
\end{equation}

For shuffle-heavy computing jobs, we can optimize the job completion time by reducing \(T_{P\_Shuffle}\). Improving IO speed is a effective way to reduce shuffle time. Another optimization method is to use the idle IO resources in the Map phase for pre-fetching(see Figure \ref{fig:model_basic}). Both of the above methods can effectively reduce \(T_{P\_Shuffle}\). When using FRQ model to describe a computation job, we can easily analyze the resource scheduling strategy of the computation framework. Different computing frameworks may use different resource scheduling strategies. FRQ model can evaluate the scheduling strategies of these computing frameworks and help us optimize them.

\subsection{Model Analysis}\label{model_analysis}

\begin{figure}
	\centering
	\begin{minipage}[hb]{\linewidth}
		\begin{subfigure}{\linewidth}
			\begin{minipage}{\linewidth}
				\includegraphics[width=\linewidth]{fig/model_original}
				\caption{\color{blue}Full Serial Mapreduce}
				\label{fig:model_original}
			\end{minipage}
			\begin{minipage}{\linewidth}
				\includegraphics[width=\linewidth]{fig/model_hadoop}
				\caption{\color{blue}Hadoop Mapreduce}
				\label{fig:model_hadoop}
			\end{minipage}
		\end{subfigure}
		\caption{\color{blue}FRQ Model With Different Scheduling Strategies}
		\label{fig:model_strategies}
	\end{minipage}
\end{figure}

FRQ model can describe a variety of resource scheduling strategies. First, we analyze a naive scheduling strategy. As shown in Figure \ref{fig:model_original}, FRQ model describes a Mapreduce job that is fully serially executed. The parallel time of shuffle phase and the Reduce phase is \(0\), in which case \(T_{P\_Shuffle}\) is \(0\). Therefore, the overhead of the Reduce phase is 0. But since shuffle and computation are serial execution, the total execution time of job becomes longer:
\begin{equation}
\label{equation_Tjob2}
\begin{aligned}
    T_{Job} &= T_{Map} + T_{Shuffle} + T_{Reduce}
\end{aligned}
\end{equation}
Obviously, this is an inefficient scheduling strategy. No computing framework uses this scheduling method. Due to serialization, the IO resource is idle during the Reduce phase and Map phase. The scheduling strategy is naive and has a lot of room for optimization.

Figure \ref{fig:model_hadoop} shows the scheduling strategy of Hadoop Mapreduce. In Hadoop Mapreduce, Shuffle phase and Reduce phase start at the same time. In this case, \(T_{P\_Shuffle}\) is equal to \(T_{Shuffle}\). Due to the increase in \(T_{P\_Shuffle}\), the time of Reduce phase will increase(see equation \ref{equation_Treduce}). Because the Shuffle phase and the computation phase are executed in parallel, the total execution time of job is the sum of \(T_{Map}\) and \(T_{Reduce}\)(see equation \ref{equation_Tjob}). The execution time of Shuffle phase is hidden in the Reduce phase. This scheduling strategy is much more efficient than the one in Figure 1. However, after analyzing this model, we found that the IO resource in the Map phase is idle. This scheduling strategy can be optimized.

\begin{figure}
	\centering
	\begin{minipage}[hb]{\linewidth}
		\begin{subfigure}{\linewidth}
			\begin{minipage}{\linewidth}
				\includegraphics[width=\linewidth]{fig/model_scache1}
				\caption{\color{blue}If \(V_{Map} \times R \ge V_{Shuffle}\)}
				\label{fig:model_scache1}
			\end{minipage}
			\begin{minipage}{\linewidth}
				\includegraphics[width=\linewidth]{fig/model_scache2}
				\caption{\color{blue}If \(V_{Map} \times R < V_{Shuffle}\)}
				\label{fig:model_scache2}
			\end{minipage}
		\end{subfigure}
		\caption{\color{blue}FRQ Model With SCache}
		\label{fig:model_scache}
	\end{minipage}
\end{figure}

Figure \ref{fig:model_scache} shows the scheduling strategy for Hadoop Mapreduce with SCache (Suppose N is 4). SCache starts pre-fetching and pre-scheduling in the Map phase. This scheduling strategy can make better use of resources and avoid IO resources being idle in the Map phase. According to the design of SCache pre-fetching, we found that using FRQ model to describe the scheduling strategy of SCache needs to distinguish two situations:

\begin{itemize}
    \item 
    \(V_{Map} \times R \ge V_{Shuffle}\)(Figure \ref{fig:model_scache1}): 
	The meaning of \(V_{Map} \times R\) is the speed at which shuffle data is generated. The meaning of the inequality is that the speed of generating shuffle data(\(V_{Map} \times R\)) is greater than or equal to the shuffle speed(\(V_{Shuffle}\)). When the Round1 of the Map phase ends, the SCache starts shuffling data until the end of the shuffle phase. Due to shuffle speed is slower, the shuffle phase is uninterrupted. SCache transmit the shuffle data generated in the last round of Map phase during the Reduce phase. Therefore \(T_{PShuffle}\) is equal to one-\(N\) of the total time of the shuffle phase:
	\begin{equation}
		\label{equation_Tpshuffle1}
		\begin{aligned}
			T_{P\_Shuffle} &= T_{Shuffle} - \frac{(N - 1)\times T_{Map}}{N}
		\end{aligned}
	\end{equation}
	
    \item \(V_{Map} \times R < V_{Shuffle}\)(Figure \ref{fig:model_scache2}): 
	When the speed of generating shuffle data(\(V_{Map} \times R\)) is less than the shuffle speed(\(V_{Shuffle}\)), SCache needs to wait for shuffle data to be generated. As Figure \ref{fig:model_scache2} shown, the shuffle phase will be interrupted in each Round. Thus \(T_{P_Shuffle}\) is equal to the total time of shuffle (\(T_{Shuffle}\)) minus the time that shuffle is executed in the Map phase:
	\begin{equation}
		\label{equation_Tpshuffle2}
		\begin{aligned}
			T_{P\_Shuffle} &= T_{Shuffle} \times \frac{1}{N}
		\end{aligned}
	\end{equation}
\end{itemize}

% After analyzing the above two situations, we figure out the formula for \(T_{P\_Shuffle}\):
% \begin{equation}
%     \label{equation_Tpshuffle}
%     \begin{aligned}
%     T_{P\_Shuffle} &=
%         \begin{cases} 
%         T_{Shuffle} - \frac{(N - 1)\times T_{Map}}{N} & , V_{Map} \times R \ge V_{Shuffle} \\ 
%         T_{Shuffle} \times \frac{1}{N} & , V_{Map} \times R < V_{Shuffle}
%         \end{cases}
%     \end{aligned}
% \end{equation}
% If the shuffle speed is slow(\(V_{Map} \times R \ge V_{Shuffle}\)), SCache transmit the shuffle data generated in the last round of Map phase during the Reduce phase. Therefore \(T_{PShuffle}\) is equal to one-\(N\) of the total time of the shuffle phase.

% If the shuffle speed is fast (\(V_{Map} \times R < V_{Shuffle}\)), \(T_{P_Shuffle}\) is equal to the total time of shuffle (\(T_{Shuffle}\)) minus the time that shuffle is executed in the Map phase.

Compared to the original Hadoop Mapreduce resource scheduling strategy, Hadoop Mapreduce with SCache reduces \(T_{P\_Shuffle}\) and thus lessens \textit{Reduce Time}(\(T_{Reduce}\)). This is how pre-fetching optimizes the total execution time of job.


\begin{table*}[!t]
\renewcommand{\arraystretch}{1.3}
\caption{\color{blue}Hadoop Mapreduce on 4 nodes cluster in FRQ model}
\label{table1}
\centering
\(D\): GB, \(V_{i}\): GB/s, \(T_{i}\): s
\begin{tabular}{|c||c|c|c|c|c|c|c||c|c|c|c|c|c|c|}
\hline
 &
\(D\) &	
\(R\) &	
\(N\) &	
\(V_{Map}\) &	
\(V_{Reduce}\) &	
\(V_{Shuffle}\) &	
\(K\) &	
\(T_{Map}\) &	
\(T_{Shuffle}\) &	
\(T_{P\_Shuffle}\) &
\(T_{Reduce}\) & 
\(T_{Job}\) & 
\(Exp T_{Job}\) &
\(Error\)\\

\hline
 & 16	& 1	& 2 &	0.65 &	1 &	0.47 &	0.5 &	24.62 &		34.04	 &	21.73 &	26.87 &	51.48	& 55  &		6.39\% \\
 SCache
 & 32	& 1	& 4 &	0.65 &	1 &	0.47 &	0.5 &	49.23 &		68.09	 &	31.16 &	47.58 &	96.81	& 104 & 	6.91\% \\
 & 48	& 1	& 6 &	0.65 &	1 &	0.47 &	0.5 &	73.85 &		102.13 &	40.59 &	68.29 &	142.14	& 151 & 	5.87\% \\
 & 64	& 1	& 8 &	0.65 &	1 &	0.47 &	0.5 &	98.46 &		136.17 &	50.02 &	89.01 &	187.47	& 193 & 	2.87\% \\
 \hline
 & 16	& 1 & 2 &	0.65 &	1 &	0.47 &	0.6 &	24.62 &		34.04	&	34.04	&	36.43	&	61.04	&	73	&	16.38\%	\\
 Legacy
 & 32	& 1 & 4 &	0.65 &	1 &	0.47 &	0.6 &	49.23 &		68.09	&	68.09	&	72.85	&	122.08	&	135	&	9.57\%	\\
 & 48	& 1 & 6 &	0.65 &	1 &	0.47 &	0.6 &	73.85 &		102.13	&	102.13	&	109.28	&	183.12	&	188	&	2.59\%	\\
 & 64	& 1 & 8 &	0.65 &	1 &	0.47 &	0.6 &	98.46 &		136.17	&	136.17	&	145.70	&	244.16	&	249	&	1.94\%	\\
\hline
\end{tabular}
\end{table*}
% \subsection{Performance Analysis}\label{model_performance_analysis}

\begin{table*}[!t]

\renewcommand{\arraystretch}{1.3}
\caption{\color{blue}Hadoop Mapreduce on 50 AWS m4.xlarge nodes cluster in FRQ model}
\label{table2}
\centering
\(D\): GB, \(V_{i}\): GB/s, \(T_{i}\): s
\begin{tabular}{|c||c|c|c|c|c|c|c||c|c|c|c|c|c|c|}
\hline
 &
\(D\) &	
\(R\) &	
\(N\) &	
\(V_{Map}\) &	
\(V_{Reduce}\) &	
\(V_{Shuffle}\) &	
\(K\) &	
\(T_{Map}\) &	
\(T_{Shuffle}\) &	
\(T_{P\_Shuffle}\) &
\(T_{Reduce}\) & 
\(T_{Job}\) & 
\(Exp T_{Job}\) &
\(Error\)\\

\hline
& 128	& 1 & 5 &	1.15 &	1.46	&	1.4 &	0.5 &	111.30 &	91.43	&	18.29	&	96.81	&	208.12	&	232	&	10.29\%	\\
SCache
& 256	& 1 & 5 &	1.15 &	1.46	&	1.4 &	0.5 &	222.61 &	182.86	&	36.57	&	193.63	&	416.24	&	432	&	3.65\%	\\
& 384	& 1 & 5 &	1.15 &	1.46	&	1.4 &	0.5 &	333.91 &	274.29	&	54.86	&	290.44	&	624.36	&	685 &	8.85\%	\\
% & 512	& 1 & 5 &	1.15 &	1.46	&	1.4 &	0.5 &	445.22 &	365.71	&	91.43	&			&	841.62	&	1135 &	25.85\%	\\
\hline
& 128	& 1 & 5 &	1.15 &	1.46	&	1.4 &	0.6 &	111.30 &	91.43	&	91.43	&	142.53	&	253.83	&	266 &	4.57\%	\\
Legacy
& 256	& 1 & 5 &	1.15 &	1.46	&	1.4 &	0.6 &	222.61 &	182.86	&	182.86	&	285.06	&	507.67	&	524 &	3.12\%	\\
& 384	& 1 & 5 &	1.15 &	1.46	&	1.4 &	0.6 &	333.91 &	274.29	&	274.29	&	427.59	&	761.50	&	776 &	1.87\%	\\
% & 512	& 1 & 5 &	1.15 &	1.46	&	1.4 &	0.6 &	445.22 &	365.71	&	365.71	&	588.40	&	1033.62	&	1312 &	21.22\%	\\

\hline
\end{tabular}
\end{table*}

\subsection{Model Verification}\label{model_verification}
In order to verify FRQ model, we run experiment on two environments. The first environment is on Amazon EC2 and it has 50 m4.xlarge nodes as shown in Section \ref{stepup}. Another environment is in our lab. Out lab environment has 4 nodes and each nodes has 128GB and 32 CPUs. To simplify the calculation of the FRQ model, we use Hadoop Mapreduce as framwork(only three phases: \textit{Map, Shuffle,} and \textit{Reduce}) and Terasort as experimental application. We deploye Hadoop with SCache and without SCache on both environments.

Table \ref{table1} shows the calculational results of FRQ model in the lab environment. Workload is from 16 GB to 64 GB. \(D\) and \(N\) are set according to the application parameters. \(R, V_{Map}, V_{Shuffle},\)and \(V_{Reduce}\) are calculated based on experimental results. K is the empirical value, we set \(K\) to 0.5 and 0.6, which reflects that \(T_{P\_Shuffle}\) has less impact on the Reduce phase in the case of SCache. The formulas of \(T_{Job}, T_{Map}, T_{Reduce}\) and \(T_{Shuffle}\) are Equation \ref{equation_Tjob}, Equation \ref{equation_Tmap}, Equation \ref{equation_Treduce} and Equation \ref{equation_Tshuffle}, respectively. In the case of SCache, Terasort on Hadoop Mapreduce satisfies the situation in Figure \ref{fig:model_scache1}(\(V_{Map} \times R \ge V_{Shuffle}\)), thus the formula of \(T_{P\_Shuffle}\) is Equation \ref{equation_Tpshuffle1}. In the case of Legacy, since pre-fetching is not used, \(T_{P\_Shuffle}\) is equal to \(T_{Shuffle}\)(see Equation \ref{equation_Tshuffle}). \(ExpT_{Job}\) represents the actual experiment data, we calculate \(Error\) according to \(T_{Job}\) and \(ExpT_{Job}\). The formular of \(Error\) is:
\begin{equation}
	\label{equation_error}
	\begin{aligned}
		Error &= \frac{ExpT_{Job} - T_{Job}}{T_{Job}}
	\end{aligned}
\end{equation}

Table \ref{table2} shows the calculational results of FRQ model in Amazon EC2 environment. \(V_{Map}, V_{Shuffle},\) and \(V_{Reduce}\) are modified because of the different hardware devices. We also set K to the same empirical value. The formulas in the table are all the same except \(T_{P\_Shuffle}\). In this environment, Terasort on Hadoop Mapreduce satisfies the situation in Figure \ref{fig:model_scache2}(\(V_{Map} \times R < V_{Shuffle}\)), thus the formula of \(T_{P\_Shuffle}\) is Equation \ref{equation_Tpshuffle2}. In the previous case, \(T_{P\_Shuffle}\) is still euqal to \(T_{Shuffle}\).

\begin{figure}
	\centering
	\begin{minipage}[hb]{\linewidth}
		\begin{subfigure}{\linewidth}
			\begin{minipage}{\linewidth}
				\includegraphics[width=\linewidth]{fig/hadoop_net1}
				\caption{\color{blue}If \(V_{Map} \times R \ge V_{Shuffle}\)}
				\label{fig:hadoop_net1}
			\end{minipage}
			\begin{minipage}{\linewidth}
				\includegraphics[width=\linewidth]{fig/hadoop_net2}
				\caption{\color{blue}If \(V_{Map} \times R < V_{Shuffle}\)}
				\label{fig:hadoop_net2}
			\end{minipage}
		\end{subfigure}
		\caption{\color{blue}Network utilization on Hadoop Mapreduce with SCache}
		\label{fig:hadoop_net}
	\end{minipage}
\end{figure}

In order to verify the above-mentioned two cases when using SCache, we monitor network utilization and plot it in Figure \ref{fig:hadoop_net}. Figure \ref{fig:hadoop_net1} shows the utilization of Terasort in the lab environment. The network utilization remains high until shuffle phase is complete. This situation is consistent with Figure \ref{fig:model_scache1}. Figure \ref{fig:hadoop_net2} shows the utilization in Amazon EC2 environment. The network utilization is 5 regular peaks, this situation is also consistent with Figure \ref{fig:model_scache2}. Therefore, we believe that FRQ model is able to accurately describe framework with SCache.

In terms of accuracy, the experimental values are all greater than the calculated values. This is because the application has some extra overhead at runtime, such as network warm-up, the overhead of allocating slots, and so on. In the case where the input data is small and the total time is short, the error caused by the overhead is amplified. Overall, the error between \(T_{Job}\) and \(ExpT_{Job}\) is basically below 10\%, such errors are within tolerance . Therefore, We believe that FRQ model can accurately describe DAG framework.
}
\section{Evaluation}\label{evaluation}
{\color{blue}
This section reveals the evaluation of SCache with comprehensive workloads and benchmarks which include common operations in the industrial big-data analysis. 
% To show the compatibility of SCache as a cross-framework plug-in, we implement and evaluate SCache on both Spark and Hadoop MapReduce. 
We implement and evaluate SCache on Spark and Hadoop MapReduce, since they are the two most distributed computing frameworks using in industrial big-data analysis.
First, we evaluate the Spark with SCache in 3 different benchmarks.
We run a Spark job with single shuffle to analyze hardware utilization and see the impacts of different components from the scope of a task to a job. 
Then we use a recognized shuffle intensive benchmark --- Terasort to evaluate SCache with different data partition schemes.
To prove the performance gain of SCache with a industrial production workload, we also evaluate Spark TPC-DS\footnote{https://github.com/databricks/spark-sql-perf} and present the overall performance improvement.
% In order to prove the performance gain of SCache with a real production workload, we also evaluate Spark TPC-DS\footnote{https://github.com/databricks/spark-sql-perf} and present the overall performance improvement.
% To prove the compatibility of SCache as a cross-framework plug-in, we implemented SCache on both Hadoop MapReduce and Spark. 
Second, due to the simple DAG computing in Hadoop MapReduce, we only use Terasort as a shuffle-heavy benchmark to evaluate the performance of Hadoop MapReduce with SCache.
}
Finally, we measure the overhead of weighted reservoir sampling. 

In summary, SCache can decrease ~$89\%$ time of Spark shuffle without introducing extra network transfer.
More impressively, the overall completion time of TPC-DS can be improved ~$40\%$ on average by applying the optimization from SCache.
{\color{blue}Meanwhile, Hadoop MapReduce with SCache optimizes job completion time by up to $15\%$ and an average of $13\%$}

% Because a complex Spark application consists of multiple stages. The completion time of each stage varies under different input data, configurations and different number of stages. This uncertainty leads to the dilemma that dramatic fluctuation occurs in overall performance comparison. To present a straightforward illustration, we limit the scope of most evaluations in a single stage.
\begin{figure*}
	\centering
	\begin{minipage}[t]{.32\linewidth}
		\begin{subfigure}{\linewidth}
			\begin{minipage}{\linewidth}
				\includegraphics[width=\linewidth]{fig/groupbymapstage}
				\caption{Map Stage Completion Time}
				\label{fig:mapstage}
			\end{minipage}
			\begin{minipage}{\linewidth}
				\includegraphics[width=\linewidth]{fig/groupbyreducestage}
				\caption{Reduce Stage Completion Time\newline}
				\label{fig:reducestage}	
			\end{minipage}
		\end{subfigure}
		\vspace{-1em}
		\caption{Stage Completion Time of\newline Single Shuffle Test}
		\label{fig:singleshuffle}
	\end{minipage}	
	\begin{minipage}[t]{.32\linewidth}
		\begin{subfigure}{\linewidth}
			\begin{minipage}{\linewidth}
				\includegraphics[width=\linewidth]{fig/groupbymaptask}
				\caption{Median Task in Map Stages}
				\label{fig:maptask}
			\end{minipage}

			\begin{minipage}{\linewidth}
				\includegraphics[width=\linewidth]{fig/groupbyreducetask}
				\caption{Median Task in Reduce Stages\newline}
				\label{fig:reducetask}
			\end{minipage}
		\end{subfigure}
		\vspace{-1em}
		\caption{Median Task Completion\newline Time of Single Shuffle Test}
		\label{fig:singleshuffletask}
	\end{minipage}	
	\begin{minipage}[t]{.32\linewidth}
		\begin{subfigure}{\linewidth}
			\vspace{-0.5em}
			\begin{minipage}{\linewidth}
				\includegraphics[width=\linewidth]{fig/tera}
				\caption{Reduce Stage of First Shuffle}
				\label{fig:terasort}
			\end{minipage}

			\begin{minipage}{\linewidth}
				\vspace{0.5em}
				\includegraphics[width=\linewidth]{fig/tera_shuffle}
				\caption{Network Traffic of Second Shuffle}
				\label{fig:terashuffle}
			\end{minipage}
		\end{subfigure}
		% \vspace{0.1em}
		\caption{Terasort Evaluation}
	\end{minipage}
\end{figure*}

\subsection{Setup}\label{stepup}
We modified Spark to enable shuffle optimization of SCache as a representative.
The shuffle configuration of Spark is set to the default\footnote{http://spark.apache.org/docs/1.6.2/configuration.html}. 
We run the experiments on a 50-node m4.xlarge cluster on Amazon EC2\footnote{http://aws.amazon.com/ec2/}. 
Each node has 16GB memory and 4 CPUs. The network bandwidth provided by Amazon is insufficient. 
Our evaluations reveal the bandwidth is only about 300 Mbps (see Figure \ref{fig:util}).

\subsection{Spark with SCache}\label{sparkscache}

\subsubsection{Simple DAG Analysis}\label{simpledag}
%\subsubsection{Hardware Utilization}
We first run the same single shuffle test shown in Figure \ref{fig:util}. 
For each stage, we run 5 rounds of tasks with different input size. 
As shown in Figure \ref{fig:scache_util}, the hardware utilization is captured from one node during the job. 
Note that since the completion time of whole job is about $50\%$ less than Spark without SCache, the duration of Figure \ref{fig:scache_util} is cut in half as well. 
An overlap among CPU, disk, and network can be easily observed in Figure \ref{fig:scache_util}. 
It is because the decoupling of shuffle prevents the computing resource from being blocked by I/O operations. 
On the one hand, the decoupling of shuffle write helps free the slot earlier, so that it can be re-scheduled to a new map task.
On the other hand, with the help of shuffle pre-fetching, the decoupling of shuffle read significantly decreases the CPU idle time at the beginning of a reduce task.
At the same time, SCache manages the hardware resources to store and transfer shuffle data without interrupting the computing process.
As a result, the utilization and multiplexing of hardware resource are increased, thus improving the performance of Spark. 
The performance evaluation in Figure \ref{fig:singleshuffletask} shows the consistent results with our observation of hardware utilization. 
% By running Spark with SCache, the completion time of map stage can be reduced $10\%$ on average. 
% For reduce stage, instead, SCache achieves a ~$75\%$ performance gain in the completion time of the reduce stage.
% A detail analysis into the nutshell of varied overall performance gain on different stages is presented with Figure \ref{fig:singleshuffletask}. 
For each stage, we pick the task that has median completion time. 
In the map task, the disk operations are replaced by the memory copies to decouple the shuffle write. 
It helps eliminate $40\%$ of shuffle write time (Figure \ref{fig:maptask}), which leads to a $10\%$ improvement of map stage completion time in Figure \ref{fig:mapstage}. 
Note that the shuffle write time can be observed even with the optimization of SCache. 
The reason is that before moving data out of Spark's JVM, the serialization is inevitable and CPU intensive \cite{makingsense}. 

In the reduce task, most of the shuffle overhead is introduced by network transfer delay. 
By doing shuffle data pre-fetching based on the pre-scheduling results, the explicit network transfer is perfectly overlapped in the map stage. 
With the help of the co-scheduling scheme, SCache guarantees that each reduce task has the benefit of shuffle pre-fetching. 
The in-memory cache of shuffle data further reduces the shuffle read time. 
As a result, the combination of these optimizations decreases ~$100\%$ overhead of the shuffle read in a reduce task (Figure \ref{fig:reducetask}). 
In addition, the heuristic algorithm can achieve a balanced pre-scheduling result, thus providing ~$80\%$ improvement in reduce stage completion time (Figure \ref{fig:reducestage}).

Overall, SCache can help Spark decrease by ~$89\%$ overhead of the whole shuffle process. 

\subsubsection{Terasort}
We also evaluate Terasort\cite{terasort} --- a recognized shuffle intensive benchmark for distributed system analysis. 
Terasort consists of two consecutive shuffles. 
The first shuffle reads the input data and uses a hash partition function for re-partitioning. 
As shown in Figure \ref{fig:terasort}, Spark with SCache runs 2 $\times$ faster during the reduce stage of the first shuffle, which is consistent with the results in Section \ref{simpledag}. 
It further proves the effectiveness of SCache's optimization.

The second shuffle of Terasort partitions the data through a Spark RangePartitioner. 
% As the range bounds set by range partitioner almost match the same pattern of the first shuffle, almost $93\%$ of input data is from one particular map task for each reduce task. So we take the second shuffle as an extreme case to evaluate the scheduling locality for SCache.
In the second shuffle, almost $93\%$ of input data of a reduce task is produced by one particular map task. 
So we take the second shuffle as an extreme case to evaluate the heuristic locality swap of SCache.
In this shuffle, Spark schedules a reduce task to the node that produces most input data. 
By doing this, Spark minimizes the shuffle data through network. 
At the same time, Figure \ref{fig:terashuffle} reveals that SCache produces exactly same network traffic as Spark. 
It implies that the heuristic locality swap of SCache can obtain the best locality while balancing the load. 
\begin{figure*}
	\centering
	\includegraphics[width=.9\textwidth]{fig/tpcds}
	\caption{TPC-DS Benchmark Evaluation}
	\label{fig:tpcds}
	\vspace{-1em}
\end{figure*}

\subsubsection{Industrial Production Workload}
We also evaluate some queries from TPC-DS\footnote{http://www.tpc.org/tpcds/}. 
TPC-DS benchmark is designed for modeling multiple users submitting varied queries (e.g. ad-hoc, interactive OLAP, data mining, etc.). 
TPC-DS contains 99 queries and is considered as the standardized industry benchmark for testing big data systems. 
% We evaluate the performance of Spark with SCache by picking some of the TPC-DS queries with shuffle intensive attribute. 
As shown in Figure \ref{fig:tpcds}, the horizontal axis is query name and the vertical axis is query completion time. 
Note that we skip some queries due to the compatible issues. 
Spark with SCache outperforms the original Spark in almost all tested queries. 
Furthermore, in many queries, Spark with SCache outperforms original Spark by an order of magnitude. 
It is because that those queries contain shuffle-heavy operations such as \textit{groupby}, \textit{union}, etc.
The overall reduction portion of query time that SCache achieved is $40\%$ on average. 
Since this evaluation presents the overall job completion time of queries, we believe that our shuffle optimization is promising.

\begin{figure*}
	\centering
	\begin{subfigure}{.47\linewidth}
		\includegraphics[width=\linewidth]{fig/hadoop_terasort_scache}
		\caption{\color{blue}Hadoop MapReduce with SCache}
		\label{fig:hadoop_terasort_scache}
	\end{subfigure}
	\begin{subfigure}{.47\linewidth}
		\includegraphics[width=\linewidth]{fig/hadoop_terasort_origin}
		\caption{\color{blue}Hadoop MapReduce without SCache}
		\label{fig:hadoop_terasort_origin}
	\end{subfigure}
	\caption{\color{blue}CPU utilization and I/O throughput of a node during a Hadoop MapReduce Terasort job}
	\label{fig:hadoop_terasort}
\end{figure*}

{\color{blue}
\subsection{Hadoop MapReduce with SCache}

To prove SCache compatibility as a cross-framework plug-in, we also implemented SCache on Hadoop MapReduce. 
Although the simple DAG computing alleviates the effect of \textit{pre-scheduling}, the shuffle-heavy jobs can still be optimized by the SCache shuffle data management.

% Figure \ref{fig:hadoop_terasort} shows the hardware resource utilization of Hadoop MapReduce running Terasort. Both figures have the same proportion of time. 
As Figure \ref{fig:hadoop_terasort} shows, Hadoop MapReduce with SCache brings $15\%$ of total time optimization with 384GB input data size. 
As shown in Figure \ref{fig:hadoop_terasort_origin}, Hadoop MapReduce without SCache writes intermediate data locally in the map phase. The shuffle phase and the reduce phase start simultaneously. Because a large amount of shuffle data reaches the network bottleneck, the beginning part of reduce phase needs to wait for network transfer. This causes the CPU resources to be idle. 
On the other hand, in Figure \ref{fig:hadoop_terasort_scache}, Hadoop MapReduce with SCache starts pre-fetching in the map phase. This avoids the reduce phase waiting for the shuffle data. Furthermore, pre-fetching utilizes the idle I/O throughput in the map phase. As shown in Figure \ref{fig:hadoop_terasort_time}, after better fine-grained utilization of hardware resources, Hadoop MapReduce with SCache optimizes Terasort overall completion time by up to $15\%$ and an average of $13\%$ with input data sizes from 128GB to 512GB.
}

\subsection{Overhead of Sampling}
We evaluate the overhead of sampling with different input sizes and numbers of nodes. 
In Figure \ref{fig:sampling}, the overhead of sampling only grows with the increase of input size on each node, but remains relatively stable when the cluster size scales up.
Since the shuffle data is short-lived, write-once, and read-once, the central controller of SCache does not have to collect and manage complex metadata. 
Meanwhile, most of the optimizations such as fetching and storing shuffle data are finished by workers independently. 
So the cost of pre-scheduling algorithm and memory management are unlikely to make the master become the bottleneck of the scalability.
Combined with the sampling overhead evaluation, we believe that SCache is scalable.

\begin{figure}
    \centering
    \begin{minipage}{0.4\textwidth}
        \centering
        \includegraphics[width=\textwidth]{fig/hadoop_terasort_time} % first figure itself
        \caption{\color{blue}Hadoop MapReduce Terasort Completion Time}
		\label{fig:hadoop_terasort_time}
    \end{minipage}\hfill
    \begin{minipage}{0.4\textwidth}
        \centering
        \includegraphics[width=\textwidth]{fig/sampling} % second figure itself
        \caption{Sampling Overhead}
		\label{fig:sampling}
    \end{minipage}
\end{figure}
\section{Related Work}

{\color{blue}
\textbf{Modeling}: Most well-known DAG computing frameworks use similar \textit{Bulk Synchronize Parallel}(BSP)\cite{valiant1990bridging} model to control data synchronization in each computing phase, i.e. \textit{stage} in Spark, \textit{superstep} in Pregel\cite{malewicz2010pregel} and so on(Apache Hadoop Madpreduce\footnote{http://hadoop.apache.com/} can also be considered as a special case of only one superstep in BSP model). 
In the process of optimizing the shuffle phases between adjacent computing phases, we design a performance model to assist in analyzing computing process.
Inspired by \cite{verma2011aria}, \cite{herodotou2011hadoop}, and \cite{polo2010performance}, we present \textit{Framework Resources Quantification}(FRQ) model to describe the performance of DAG frameworks.
In \cite{verma2011aria}, the author designs a model and divides execution into three parts: Map, Shuffle and Reduce. And then the author uses a greedy algorithm to roughly calculate the max, min and mean execution time of Map and Reduce phases.
In \cite{herodotou2011hadoop}, the author presents a detailed set of mathematical performance models for describing the execution of a MapReduce job. The execution is seperated into the phases: Read, Map, Collect, Spill, Merge, Shuffle, Merge, and Reduce. The performance models describe each above-mentioned phases and combine into a overall Mapreduce job model. 
However, none of the above models meet our requirements for analyzing shuffle.
The model in \cite{verma2011aria} is not able to accurately describe the overhead caused by shuffle under different scheduling strategies. The model in \cite{herodotou2011hadoop} calculates overhead of various phases including shuffle, but most of them are redundant. 
Different from these models, our FRQ model quantifies computing and I/O resources and displays them in time dimension. FRQ model focuses on describing the overhead caused by shuffle in different scheduling strategies, which satisfy our demand.
}

\textbf{Pre-scheduling}: Slow-start from Apache Hadoop MapReduce is a classic approach to handle shuffle overhead. 
Starfish \cite{starfish} gets sampled data statics for self-tuning system parameters (e.g. slow-start, etc). 
DynMR \cite{dynmr} dynamically starts reduce tasks in late map stage. 
All of them have the explicit I/O time in occupied slots. 
SCache instead starts shuffle pre-fetching without consuming slots. 
iShuffle \cite{ishuffle} decouples shuffle from reducers and designs a centralized shuffle controller. 
But it can neither handle multiple shuffles nor schedule multiple rounds of reduce tasks. 
iHadoop \cite{ihadoop} aggressively pre-schedules tasks in multiple successive stages to start fetching shuffle. 
But we have proved that randomly assign tasks may hurt the overall performance in Section \ref{randomassign}. 
Different from these works, SCache pre-schedules multiple shuffles without breaking load balancing. 

%  by combining DAG information and heuristic algorithms.
\textbf{Delay-scheduling}: Delay Scheduling \cite{delay} delays tasks assignment to get better data locality, which can reduce the network traffic. 
ShuffleWatcher \cite{shufflewatcher} delays shuffle fetching when network is saturated. 
At the same time, it achieves better data locality. 
Both Quincy \cite{quincy} and Fair Scheduling \cite{preemptive} can reduce shuffle data by optimizing data locality of map tasks. 
But all of them cannot mitigate explicit I/O in both map and reduce tasks. 
In addition, their optimizations fluctuate under different network performances and data distributions, whereas SCache can provide a stable performance gain by shuffle data pre-fetching and in-memory caching.

\textbf{Network layer optimization}: Varys \cite{varys} and Aalo \cite{aalo} provide the network layer optimization for shuffle transfer. 
Though the efforts are limited throughout whole shuffle process, they can be easily applied on SCache to further improve the performance.
\section{Conclusion}
 In this paper, we present SCache, a cross-framework open source shuffle optimization system for DAG computing frameworks. SCache decouples the shuffle from computing pipeline and leverages shuffle data pre-fetching to mitigate the I/O overhead of the whole system. By scheduling tasks with application context, SCache bridges the gap among computing stages. 
%  {\color{red}
%  Our implementation with Spark and evaluations show that SCache can provide a promising speedup to the DAG framework. 
%  }
 {\color{blue}
 At the expense of inserting hundred lines of code, we adapt Spark and Hadoop Mapreduce to SCache. Our evaluation result shows that SCache can provide a promising speedup to the DAG framework. 
 Furthermore, we propose \textit{Framework Resources Quantification}(FRQ) model to assist in analyzing shuffle process of DAG computing frameworks. At last we use FRQ model to verify SCache shuffle optimization by mathematics.
 Therefore we believe that SCache is a simple and efficient system to enhance the performance of most DAG computing frameworks. 
 }

\section*{\edited{Acknowledgements}}
This work was supported in part by National Key Research \& Development Program
of China (No. 2016YFB1000502), National NSF of China (NO. 61672344, 61525204,
61732010), and Shanghai Key Laboratory of Scalable Computing and Systems.


% \printcredits

% \footnotesize

% \vskip12pt

\bibliographystyle{model1-num-names}
\bibliography{biblio}

% \bio{./fig/author/Rui_Ren}
% \textbf{Rui Ren} was born in Shaanxi, China, in 1978. He received the B.S. and M.S. degrees from Shanghai Jiao Tong University, China, in 2000 and 2004, respectively. He is currently a lecturer in the School of Software, Shanghai Jiao Tong University, China.
% \endbio

% \vskip30pt

% \bio{./fig/author/Chunghsuan_Wu}
% \textbf{Zhongxuan Wu} received the BE degree from Shanghai Jiao Tong University, China, in 2017. He is currently pursuing a ME degree in Shanghai Jiao Tong University, China. His research interests mainly focus on distributed computing.
% \endbio

% \vskip30pt

% \bio{./fig/author/ZhouWang_Fu}
% \textbf{Zhouwang Fu} is a graduated Master student of Shanghai Jiao Tong University in China. He received his Bachelor and Master degree in software engineering Shanghai Jiao Tong University.
% \endbio

% \vskip30pt

% \bio{./fig/author/Tao_Song}
% \textbf{Tao Song} is currently working as a postdoc at Shanghai Jiao Tong University in China. He received his Ph.D. degree in computer science and M.Eng. degree in software engineering from Shanghai Jiao Tong University. His research interests include data center networking, cloud computing, artificial intelligence and swarm intelligence.
% \endbio

% \vskip30pt

% \bio{./fig/author/Zhengwei_Qi}
% \textbf{Zhengwei Qi} received the BEng and MEng degrees from Northwestern Polytechnical University, in 1999 and 2002, and the PhD degree from Shanghai Jiao Tong University, in 2005. Currently, he is a professor in the School of Software, Shanghai Jiao Tong University (China). His research interests include distributed computing, virtualized security, model checking, program analysis and embedded systems.
% \endbio

% \vskip30pt

% \bio{./fig/author/Haibing_Guan}
% \textbf{Haibing Guan} received the PhD degree from Tongji University, in 1999. He is a professor of School of Electronic, Information and Electronic Engineering, Shanghai Jiao Tong University, and the director of the Shanghai Key Laboratory of Scalable Computing and Systems. His research interests include distributed computing, network security, network storage, green IT, and cloud computing.
% \endbio

% \makeatletter

% \def\pct{\expandafter\@gobble\string\%}

% \immediate\write\@auxout{\pct\space This is a test line.\pct }

\end{document}

