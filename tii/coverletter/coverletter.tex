\documentclass[12pt,reqno]{amsart}
\setlength{\hoffset}{-.5in}
\setlength{\voffset}{-.25in}
\usepackage{amssymb,latexsym}
\usepackage{graphicx}
\usepackage{fancyhdr}
\usepackage{cancel}
\usepackage{url}
\usepackage{amssymb,amsmath,psfrag}
\usepackage{enumerate}
\newcommand\BD{\mathrm{B}}
\newcommand\SD{\mathrm{S}}
\textwidth=6.175in
\textheight=8.5in

\theoremstyle{plain}
\numberwithin{equation}{section}
\newtheorem{thm}{Theorem}[section]
\newtheorem{theorem}[thm]{Theorem}

\theoremstyle{plain}
\numberwithin{equation}{section}


\title{Summary of Difference}

% \author{Rui Ren}
% \author{Chunghsuan Wu}
% \author{Zhouwang Fu}
% \author{Tao Song}
% \author{Yanqiang Liu}
% \author{Zhengwei Qi}
% \author{Haibing Guan}


\begin{document}


\maketitle

% \section{Summary}


This paper is the extension of our conference paper at PPoPP '18 \cite{fu2018efficient}. 
% In the manuscript we uploaded, we highlight the new content in blue.
% We believe that shuffle-heavy jobs are also existed widely in industrial big-data analysis.
% SCache can also provide considerable optimization in industrial big-data.
In this paper, we combine the characteristic of the industrial big-data and DAG computing frameworks.
We provide a new data processing solution for the industrial big-data based on our conference paper.
Besides, we use the credible benchmark to demonstrate that our solution is effective in the industrial big-data analysis.
This paper makes the following distinct contributions compared with the conference paper.

\begin{enumerate}
\item 
We propose a new performance model called \textit{Framework Resources Quantification} (FRQ) model.
The FRQ model quantifies computing and I/O resources by five input parameters. 
Furthermore, the FRQ model visualizes the resources scheduling strategies of the DAG frameworks in the time dimension. 
In this paper, we use the model to evaluate the deficiencies of different resources scheduling strategies and then optimize them.
\item 
In the conference paper, we only implement SCache on Spark. 
In this paper, we also implement SCache on Hadoop MapReduce, since Spark and Hadoop MapReduce are the two most distributed computing frameworks using in industrial big-data analysis.
% By using the FRQ model to analyze Hadoop MapReduce, we can figure out that SCache is feasible to optimize the shuffle phases. 
% Meanwhile, our experiments are also consistent with the analysis of the model.
This result also demonstrates the compatibility and adaptability of SCache as a cross-framework plug-in.
% To demonstrate the compatibility and adaptability of SCache as a cross-framework plug-in, we also extended SCache on Hadoop MapReduce. 
\item
We append two parts to the evaluation section. 
Firstly, we evaluate the FRQ model in both our in-house environment and Amazon EC2 environment. The error between the FRQ’s calculated value and the experimental value is mainly below $10\%$. 
Secondly, we evaluate the performance of Hadoop MapReduce with SCache. We use the same Amazon EC2 environment as the conference paper (50-nodes m4.xlarge cluster).
According to the experiments, Hadoop MapReduce with SCache optimizes job completion time by up to $15\%$ and an average of $13\%$.
\end{enumerate}


% Besides the major contributions, we also revise the paper in many places, including writing and more related work. 
In summary, we propose a new performance model to provide a theoretical basis for performance optimization and improve evaluations by adding experiments on Hadoop MapReduce. 
We believe that the added content makes a sufficient contribution to this special issue.
% We believe that the added content makes a sufficient contribution to this journal submission.

\bibliographystyle{IEEEtran}
\bibliography{coverletter}


\end{document}