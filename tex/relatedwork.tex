\section{Related Work}
This section describes related work of shuffle optimization.

{\color{blue}
\textbf{Modeling}: Most well-known DAG computing frameworks use similar \textit{Bulk Synchronize Parallel}(BSP)\cite{valiant1990bridging} model to control data synchronization in each computing phase, i.e. \textit{stage} in Spark, \textit{superstep} in Pregel\cite{malewicz2010pregel} and so on(Hadoop Madpreduce can also be considered as a special case of only one superstep in BSP model). 
In the process of optimizing the shuffle phases between adjacent computing phases, we design a performance model to assist in analyzing computing process.
Inspired by \cite{verma2011aria}, \cite{herodotou2011hadoop}, and \cite{polo2010performance}, we present \textit{Framework Resources Quantification}(FRQ) model to describe the performance of DAG frameworks.
In \cite{verma2011aria}, the author designs a model and divides execution into three parts: Map, Shuffle and Reduce. And then the author uses a greedy algorithm to roughly calculate the max, min and mean execution time of Map and Reduce phases.
In \cite{herodotou2011hadoop}, the author presents a detailed set of mathematical performance models for describing the execution of a MapReduce job. The execution is seperated into the phases: Read, Map, Collect, Spill, Merge, Shuffle, Merge, and Reduce. The performance models describe each above-mentioned phases and combine into a overall Mapreduce job model. 
However, none of the above models meet our requirements for analyzing shuffle.
The model in \cite{verma2011aria} is not able to accurately describe the overhead caused by shuffle under different scheduling strategies. The model in \cite{herodotou2011hadoop} calculates overhead of various phases including shuffle, but most of them are redundant. 
Different from these models, our FRQ model quantifies computing and I/O resources and displays them in time dimension. FRQ model focuses on describing the overhead caused by shuffle in different scheduling strategies, which satisfy our demand.
}

\textbf{Pre-scheduling}: Slow-start from Apache Hadoop MapReduce \cite{hadoop} is the most classic approach to handle shuffle overhead. Starfish \cite{starfish} gets sampled data statics for self-tuning system parameters (e.g. slow-start, map and reduce slot ratio, etc). DynMR \cite{dynmr} dynamically starts reduce tasks in late map stage to decrease the time of waiting for outputs of map tasks. All of them left the explicit I/O time in an occupied computation slot. Instead, SCache can start shuffle pre-fetching without consuming slots. iShuffle \cite{ishuffle} decouples shuffle from reducers and designs a centralized shuffle controller. But it can neither handle multiple shuffles nor schedule multiple rounds of reduce tasks. iHadoop \cite{ihadoop} aggressively pre-schedules tasks in multiple successive stages, in order to start fetching data from previous stage earlier. But we have proved that randomly assign tasks may hurt the overall performance in Section \ref{randomassign}. Different from these works, SCache pre-schedules multiple shuffles without breaking load balancing by combining DAG information and heuristic algorithms.

\textbf{Delay-scheduling}: Delay Scheduling \cite{delay} delays tasks assignment to get better data locality, which can reduce the network traffic. ShuffleWatcher \cite{shufflewatcher} delays shuffle fetching when network is saturated. At the same time, it achieves better data locality. Both Quincy \cite{quincy} and Fair Scheduling \cite{preemptive} can reduce shuffle data by optimizing data locality of map tasks. But all of them can not mitigate explicit I/O in both map and reduce tasks. In addition their optimizations fluctuate under different network performances and data distributions, whereas SCache can provide a stable performance gain by shuffle in-memory caching and pre-fetching.

\textbf{Network layer optimization}: Varys \cite{varys} and Aalo \cite{aalo} provide the network layer optimization for shuffle transfer. Though the efforts are limited throughout whole shuffle process, they can be easily applied on SCache to further improve the performance.